\chapter{Data analysis}

\section{Spark/Blink analysis}
\label{sec:spark-analysis}

Spark detection is dependent upon number of RyR opens. Blink detection
is dependent upon the recovery of local luminal $\Ca$ recovery in the
junctional-SR. For different versions, see Sect.\ref{sec:Spark_analysis_code}.

A quark is defined as the local release with a few RyR opening that
doesn't trigger a spark. To detect spark frequency, we find the ratio 
Nspark/Nquarks.

\begin{lstlisting}
flag1[1..NSFU] = 0  ! if spark start
flag2[1..NSFU] = 0  ! if blink start
flag3[1..NSFU] = 0  ! if spark rise
flag4[1..NSFU] = 0  ! if true spark
prev_RyRopen[1..NSFU] = 0
Nquarks[1..NSFU] = 0   ! a quark is the release from a few opening RyRs
Nsparks[1..NSFU] = 0
Nblinks[1..NSFU] = 0
BlinkDuration[1..NSFU] = 0
SparkDuration[1..NSFU] = 0

DO ii = 1..t_end      
   DO nn = 1, NSFU
     ! spark start
     IF (number_RyRopen[nn] .GE. 1 .AND. flag1[nn] .EQ. 0)
        flag1[nn] = 1; flag2[nn] = 1; flag3[nn] = 1;
        t1[nn] = ii; // start spark
        t2[nn] = ii; // start blink
        prev_RyRopen[nn] = number_RyRopen[nn]
     ENDIF
     ! detect quarks
     IF (number_RyR_open .EQ. 0 .AND. flag1[nn] .EQ. 1 
          .AND. flag4 .EQ. 0)
         ! count # of blinks with 1,2,3... RyR opens
         Nquarks [nn,prev_RyRopen[nn]] = NQ[nn,prev_RyRopen[nn]] + 1 
         ! reset flags
         flag1[nn] = 0; flag2[nn] = 0; flag3[nn] = 0;
     ENDIF
     ! spark peak
     IF (number_RyRopen[nn] .GT. float(nR)/1.1d0 .AND. 
          flag1[nn] .EQ. 1 .AND. flag3[nn] .EQ. 1)
        Nspark[nn] = Nspark[nn] + 1
        flag3[nn] = 0; flag4[nn] = 1;
     ENDIF
     ! spark termination 
     IF (number_RyRopen[nn] .EQ. 0 .AND. flag1[nn] .EQ. 1 .AND.
           flag4[nn] .EQ. 1)
        flag1[nn] = 0; 
        Nspark[nn] = Nspark[nn] + 1
        SparkDuration[nn] = SparkDuration[nn] + (ii-t1[n])
     ENDIF
     ! blink detection
     IF (Ca_jsr[nn] .GT. 0.99 * Ca_nsr .AND. flag2[nn] .EQ. 1 .AND.
           flag4[nn] .EQ. 1)
         flag4[nn] = 0; flag2[nn] = 0
         Nblink[nn] = Nblink[nn] + 1;
         BlinkDuration[nn] = BlinkDuration[nn] + (ii-t2[nn])
     ENDIF
   ENDDO
ENDDO

Nspark = sum(Nspark)
Nblink = sum(Nblink)
BlinkDuration = BlinkDuration / Nblink
SparkDuration = SparkDuration / Nspark
Avg_BlinkDuration = avg(BlinkDuration)
Avg_SparkDuration = avg(SparkDuration)
Std_SparkDuration = std(SparkDuration)
Std_BlinkDuration = std(BlinkDuration)

SF = Nspark/Nblink ???? what for

SparkPeaks
Avg_SparkPeak = mean(SparkPeaks)
Std_SparkPeak = std(SparkPeaks)
SFDHM = spark full-width-half-maximum
      = when cads between 
BFDHM = blink full-width-half-maximum
\end{lstlisting}

\subsection{4 Leak}
\label{sec:leak_spark-analysis}

We use \verb!spark_analysis()!.

\subsection{4 ECC}
\label{sec:ecc_spark-analysis}


We use \verb!spark_analysis_2()!	

\subsection{4 spatials}

% \section{Spark Analysis}
% \label{sec:spatial_spark-analysis}
 
In spatial code, there is a change to detect the peak correctly. The reason is
that the peak is higher than the value at which all RyR opening at the first
time. So, we need to use a variable to keep track of the calcium subspace, and
detect it as peak when it first decrease.
\begin{verbatim}
ALLOCATE(Ca_ds_prev_dev(NSFU), stat=iserror)
\end{verbatim}
This is done in the subroutine \verb!spark_analysis_3()! (See
Sect.\ref{sec:Spark_analysis_code}).


\subsubsection{Cmyo transient, CaF transient}
\label{sec:cmyo-caf_spatial}

The level of calcium is measured via CaF. So, we use $[\CaF]$ or F/F0 or
$F_\norm$ to compared with experimental data. The data measured from microscope
is a sphere of radius $2\mum$. 

However, if we use the cube of radius $2\mum$, we'll take extra data point, and
thus may cause the level of calculated aeraged CaF lower. A solution to choose:
\begin{enumerate}
  \item only take the data from grid point fall inside the sphere of radius
  $1\mum$. 
  \item calcualte from the cube, yet with a smaller size, i.e. $1.8\mum$.
\end{enumerate} 


\subsubsection{FWHM (Full-Width Half-Max)}
\label{sec:FWHM}
 
How to detect FWHM? 
\begin{enumerate}
  \item From the CRU location (x,y,z), Detect the time at peak t0.
   \item extract the single layer (x,y) at time t0
   \item extend on 2 directions (dx, dy), stop when (x-dx,y-dy) $<$ 1/2 peak. 
   \item return 2dx
\end{enumerate}


Other code versions can be referenced in Sect.\ref{sec:Spark_analysis_code}.

% \subparagraph{4 Spatial }
% \label{sec:spatial_spark-analysis}
% 
% We use \verb!spark_analysis_3()! and \verb!spark_analysis_3D()!, which deals
% with peak not at the first all RyR openings.


% 
% \subsection{4 Spatial with rogue RyRs}
% \label{sec:spatial_spark-analysis-rogueRyR}
% 
% We use \verb!spark_analysis_rogueRyR()!. 
% 


