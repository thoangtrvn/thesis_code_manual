\chapter{Macro file (macro.h)}

The file macro.h defines some pre-processing flags to determine what to compile
in the code. 

\section{General macro names}

\begin{enumerate}
  \item Any binary flag use the two values below

\begin{verbatim}
#define _YES 1
#define _NO 0 
\end{verbatim}

\item What GPU to use
\begin{verbatim}
#define _TESLA  1
#define _FERMI  2
#define _GPU_MODEL_ _FERMI
\end{verbatim}

\item Data I/O method
\begin{verbatim}
  ! DATA TRACKING MODE
  !   0 = no I/O
  !   1 = testing purpose
  !   2 = (spatial or non-spatial) HDF files;
  !   3 = text files (all)
  !   4,5 = spatial output
  !   6 = write out restart file and the last second
  ! NOTE: restart file is written out in all cases, except _IO_NONE
  !   7 = (non-spatial) multiple datasets/file (time-Vm, Cads, Cajsr...)
  !   8 = (spatial) HDF whole-cell
#define _IO_NONE  0
#define _IO_TEST  1
#define _IO_2_HDF  2
#define _IO_2_ASCII  3
#define _IO_2_SILO_3D 4 
#define _IO_2_SILO_SLICE 5
#define _IO_RESTART_FILE 6
#define _IO_2_HDF2 7
#define _IO_2_HDF3D 8      
      !!!!example:#define _DATAMODE_  _IO_2_SILO_SLICE
      
      ! COMPUTATIONAL MODE (work for DATAMODE=1)
      !    0 - print nothing, except the simulation
      !   -1 - print ALL (except number 4)
      !    1 - print [Ca_ds], [Ca_jsr], nRo, nLo
      !    2 - print I_dhpr_T (L-type current)
      !    3 - print fraction of channels in each of the 6 L-type state
#define _COMPMODE_  1
\end{verbatim}

\item What pseudo-random number generator (PRNG) to use
\begin{verbatim}

  ! Method for PRNG
  ! SARU = generate on GPU directly
  ! MT = generate MT on GPU directly
  ! NOTE: only use SARU for now 
#define _PRNG_SARU 1
#define _PRNG_MT 2
#define _PRNG_MTCPU 3
#define _PRNG_CURAND 4
!#define _USE_PRNG_ _PRNG_SARU
\end{verbatim}


\item Safe mode
\begin{verbatim}
      ! perform all data check
      ! (may slow, as data validation is performed)
#define _SAFE_MODE_ _NO
\end{verbatim}

\item Simulation mode
\begin{verbatim}
      ! 0 = closed-cell (no ion in/out)
      ! 1 = Voltage-clamp
      ! 2 = Current-clamp
      ! 3 = restin
#define _MODE_CLOSEDCELL 0      
#define _MODE_V_CLAMP 1
#define _MODE_I_CLAMP 2
#define _MODE_RESTING 3      
!#define  _RUNMODE_  _MODE_V_CLAMP
\end{verbatim}


\item Display mode
\begin{verbatim}
      ! 0 = no graphics
      ! 1 = use CUDA/OpenGL (only Fermi)
      ! 2 = use OpenGL
      ! 3 = use VisIt
#define _DISP_NO 0
#define _DISP_CUDA_OPENGL  1
#define _DISP_OPENGL  2
#define _DISP_VISIT    3  
#define _USE_DISPLAY_ _DISP_NO
\end{verbatim}
NOTE: Not being used 

\item Analyze what? 
\begin{verbatim}
   ! how many sparks/quarks
_ANALYZE_SPARKS_ _NO
   ! calculate integral SR flux per second
_ANALYZE_SL_CALCIUM_FLUX_ _YES
   ! calculate integral SR flux per second
_ANALYZE_SR_CALCIUM_FLUX_ _YES
   ! analyze GAIN/APD90/APD50/APD20
   !V-clamp: GAIN (INTEGRAL_GAIN, PEAK_GAIN)
   !I-clamp: APD90/APD50/APD20 
_ANALYZE_ECC_ _YES
\end{verbatim}

\end{enumerate}

\section{Macro names only in compartmental model}


To avoid confusing in modifying the macros, there are several macro.h files were
created. Each one were designed for a particular purpose. For example: 
\begin{enumerate}
  \item \verb!macro.h_Iclamp! : is for I-clamp purpose
  \item \verb!macro.h_rest! : is for resting (quiescent) simulation
\end{enumerate}
Sometimes, to avoid the booming of number of macro names, you also need to
modify the code at some part to work for that macro.h file. In the beginning of
the macro.h file, there is a description for what to change. Example: 
\verb!macro.h_blockSR! is designed to balance SL calcium flux, while blocking SR
calcium. The goal of this simulation is to find the proper value for
\verb!g_bCa!. Currently, it's very hard to balance K+, Na+ and Ca2+ at the same
times. 

\begin{enumerate}
\item Scale or not
\begin{verbatim}
      ! scaling will allow testing the code using a smaller number of CaRU
      ! that give results similar to 20,000 CaRU
#define _USE_SCALING_MODE_  _YES
\end{verbatim}

 
\end{enumerate}




\section{Macro names in spatial code}

For spatial code, we use
\begin{enumerate}
\item Analyze what? (spatial model)
\begin{verbatim}
_MODE_CHECK_BALANCING_BLOCKED_SR
_MODE_CHECK_BALANCING_BLOCKED_SL
\end{verbatim}

\item How 3D array are defined: using linear form or in 3D form
\begin{verbatim}
#define _LINEAR_FORM 1
#define _3D_FORM 2
  ! use linear form for now
!#define _MEMORY_LAYOUT_ _LINEAR_FORM
\end{verbatim}
Currently, we only support linear form.

\item How we map grid points to threads in kernel
\begin{verbatim}
  ! decide how to manipulate
#define _1THREAD_1ELEMENT 1
#define _1THREAD_1DEPTHCOLUMN 2  
!!#define _GRID_STRATEGY_ _1THREAD_1ELEMENT
\end{verbatim}
Currently, we use 1 thread process 1 grid point. A better way is to use one
thread process one column. To be supported later.


\item How we arrange CRU in the spatial grid
\begin{verbatim}
  ! 1 = try to put CRUs as all possible "valid" positions
  ! 2 = put CRUs until it reaches the specified NSFU
  ! 3 = arrange CRUs, each has one or a few rogue RyRs 
  !     (check testspatialdata_file.txt)  
#define _CRU_UNIFORM 1
#define _CRU_UPTO_NSFU 2
#define _CRU_UNIFORM_WITH_ROGUE_RYR 3
#define _CRU_NONUNIFORM 4
#define _CRU_NONUNIFORM_WITH_ROGUE_RYR 5
!!!#define _ARRANGE_CRU_ _CRU_UPTO_NSFU
\end{verbatim}

NOTE: Rename
\begin{verbatim}
_CRU_UNIFORM_WITH_SMALL_RYR --> _CRU_UNIFORM_WITH_ROGUE_RYR
\end{verbatim}

\item Simulation type
\begin{verbatim}
      ! type 1 = trigger half of the number of the CRUs
      ! type 2 = increase the opening probability  of all CRUs 
      !          (adjust luminal calcium gating function)  
#define _CAFFEINE_TYPE1 1 
#define _CAFFEINE_TYPE2 2
#define _TRIGGER_RYR_CRU 3       !open a few (about 8) for a few time-steps
#define _TRIGGER_RYR_ROGUE 4      
#define _TRIGGER_LCC 5           !open 1 LCC for 0.5ms
#define _FLASHING_CALCIUM 6      !set a region of NSR with high calcium and
                                 !releasing it
#define _TRIGGER_RYR_2ADJACENT_CRUS ! use for macrospark simulation
    !the goal is to show that macrospark can only be the result of coactivation
    of two adjacent CRUs, not the calcium diffusion
    ! here we trigger CRU=3, and then CRU=4 given that CRU3 and CRU4 are
    !adjacent to each other on adjacent Z-line
#define _TRIGGER_RYR_ROGUE_MIDDLE 8  !use with trigger_rogue_RYR_4()
    ! assuming each CRU has 3 rogueRYR clusters, it trigger the one in the
    ! center to see how it can trigger the other ones, including the CRU
    
      !#define _EMULATE_CONDITION_ _CAFFEINE_TYPE1
\end{verbatim}
In case we choose \verb!_TRIGGER_RYR_CRU! or \verb!_TRIGGER_LCC!, we have a
sub-option to choose triggering one or all CRUs (added Sep,21,2012)
\begin{verbatim}
      ! use with _NUM_CRU_2_TRIGGER
      ! in case we use _TRIGGER_RYR_CRU, we can choose to trigger all
      ! or only CRU=3 only
#define _CRU3_TRIGGERED 1
#define _CRUS_ALL_TRIGGERED 2
\end{verbatim}


\item Boundary condition
\begin{verbatim}
#define _NEUMANN_COND 2
#define _DIRICHLET_COND 1
#define _ROBIN_COND 3
!#define _BOUNDARY_CONDITION_ _NEUMANN_COND      
\end{verbatim}

\item The reference volume

\begin{verbatim}
#define _V_MYO_TOTAL_REALCELL 1
#define _V_MYO_TOTAL_TESTCELL 2            
#define _V_MYO_GRIDPOINT 3
!#define _MAP_CONCENTRATION_ _V_MYO_GRIDPOINT

_V_MYO_TOTAL_REALCELL maps concentration to the real cell volume 
  of rat ventricular myocyte, i.e. 36pL
_V_MYO_TOTAL_TESTCELL maps concentration to the cell volume
  being used in the simulation
_V_MYO_GRIDPOINT concentration to the single grid point, e.g.
  0.2^3 (um^3)
\end{verbatim}
It's default to use \verb!_V_MYO_GRIDPOINT!.


\end{enumerate}

\section{Removed macro names}

NO LONGER used macro
\begin{enumerate}
  \item Moved to the \verb!testspatialdata_file.txt!
  \begin{verbatim}
  #define _0_SMALL_RYRS 1
#define _1_SMALL_RYRS 2
#define _3_SMALL_RYRS 3
#define _6_SMALL_RYRS 4
!#define _ADD_SMALL_RYRS_ _3_SMALL_RYRS
  \end{verbatim}
\end{enumerate}