\chapter{Spatial model with peripheral RyR clusters}
\label{chap:small_RyR_cluster}
% \section{Small RyR cluter}
% \label{sec:small_RyR_cluster}

In cardiac cells, it was observed a small population of ``rogue'' RyR cluster or
non-junctional RyR cluster. The size of these clusters are smaller compared to
those in the dyadic subspace. These rogue RyR clusters are believed to
facilitate the propagation of spark-induced spark. So, it's important to
investigate its role in a spatial-temporal model. This cluster has no apposed
LCC cluter. Currently, the dynamics of these RyR2 channels are assumed the same
as that in the dyadic subspace (\textcolor{red}{so we use the same
rate-transition matrices}). However, some experimental data shown that the
activities of these channels are more often. Thus, we may need to use a
different kinetic scheme for channels in the non-junctional cluster (to be
investigated later).
% \begin{verbatim} !! transition matrices for small RyR clusters (no opposing
% DHPR) REAL(KIND=dp), DIMENSION(:,:), ALLOCATABLE:: aKm2_small, aKp2_small
% INTEGER, DIMENSION(:), ALLOCATABLE :: U_Ro_small INTEGER :: irow_R_small
% \end{verbatim}

\section{New data}

OBSOLETE:
\begin{verbatim}
REAL(KIND=dp), DIMENSION(:,:), ALLOCATABLE:: compK_Rm_small, compK_Rp_small
INTEGER, DIMENSION(:,:), ALLOCATABLE :: idx_Rm_small, idx_Rp_small
INTEGER, DIMENSION(:), ALLOCATABLE :: U_Ro_small
INTEGER :: irow_R_small
INTEGER :: maxnkRm_small, maxnkRp_small  
\end{verbatim}
REPLACED BY
\begin{verbatim}
  !+rogue RyR
  REAL(KIND=dp), DIMENSION(:,:), ALLOCATABLE :: ryr_compKfree_rogue, ryr_compKcal_act_rogue
  INTEGER, DIMENSION(:,:), ALLOCATABLE :: ryr_indxKfree_rogue, ryr_indxKcal_act_rogue
  INTEGER:: ryr_Ncol_Kfree_rogue, ryr_Ncol_Kcal_act_rogue

  !+ vector tells number of open LCC/RyR in each homogeneous cluster space
  REAL(KIND=dp), DIMENSION(:), ALLOCATABLE :: U_Ro_rogue
  !+ number of rows in each vector above
  INTEGER :: irow_R_rogue
\end{verbatim}

For GPU data, we declare

OBSOLETE
\begin{verbatim}
!!for small cluster:
REAL(KIND=dp), DIMENSION(:,:), ALLOCATABLE, device :: compK_Rm_small_dev, compK_Rp_small_dev
INTEGER, DIMENSION(:, :), ALLOCATABLE, device :: idx_Rm_small_dev, idx_Rp_small_dev
REAL(KIND=dp), DIMENSION(:), ALLOCATABLE, device :: U_Ro_small_dev

ALLOCATE(compK_Rm_small_dev(0:maxnkRm_small, irow_R_small), &
       compK_Rp_small_dev(0:maxnkRp_small, irow_R_small), STAT=iserror)
ALLOCATE(idx_Rm_small_dev(0:maxnkRm_small, irow_R_small), &
         idx_Rp_small_dev(0:maxnkRp_small, irow_R_small) )
         
 ALLOCATE(U_Ro_small_dev(irow_R_small), stat=iserror)         
\end{verbatim}
REPLACED BY
\begin{verbatim}
  !!for rogue cluster:
  REAL(KIND=dp), DIMENSION(:,:), ALLOCATABLE, device :: ryr_compKfree_rogue_dev, ryr_compKcal_act_rogue_dev
  INTEGER, DIMENSION(:, :), ALLOCATABLE, device :: ryr_indxKfree_rogue_dev, ryr_indxKcal_act_rogue_dev
  REAL(KIND=dp), DIMENSION(:), ALLOCATABLE, device :: U_Ro_rogue_dev
\end{verbatim}


To generate these matrices, in the main.f95 file, we call
% \begin{verbatim}
%   !! generate transition matrices for small size RyR (no apposing DHPR)
%   CALL makeKs_RyRsmall_2(N_R_small, aKm2_small, aKp2_small, U_Ro_small, irow_R_small)
% \end{verbatim}
\begin{verbatim}
CALL makecompKs_RYRsmall_T(N_R_small, compK_Rm_small, compK_Rp_small,& 
           idx_Rm_small, idx_Rp_small, &
           U_Ro_small, irow_R_small, maxnkRm_small, maxnkRp_small, &
           setupRyR_2state_T) !!! or
CALL makecompKs_RYRsmall(N_R_small, compK_Rm_small, compK_Rp_small, &
           idx_Rm_small, idx_Rp_small, &
           U_Ro_small, irow_R_small, maxnkRm_small, maxnkRp_small, & 
           setupRyR_2state)
\end{verbatim}
and then assign
\begin{verbatim}
    compK_Rm_small_dev = compK_Rm_small
    compK_Rp_small_dev = compK_Rp_small
    idx_Rm_small_dev = idx_Rm_small
    idx_Rp_small_dev = idx_Rp_small
    U_Ro_small_dev = U_Ro_small
\end{verbatim}


%NOTE: We don't need to use compact form as RyR we are using has only 2-state. 

With small size RyR clusters, what need to be changed? 
\section{Size of data array to be increased}
\label{sec:size_data_change}

NOTE: The zero-th entry is important in device kernel for efficient calculation.
In host code, we don't need to use this.

\subsection{Ver 1}
\label{sev:ver_1}

If we assume the peripheral RyR clusters also see a subspace volume, and a
junctional SR, then we need
\begin{enumerate}
  \item number of release sites: though 1\ldots NSFU (regular),
  NSFU+1\ldots (k+1)*NSFU (small size RyR cluster) with k =
  \verb!num_smallRyRneighbors! (1, 3 or 6).
\begin{verbatim}
    ALLOCATE(Ca_ds((num_smallRyRneighbors+1)*NSFU), &
         Ca_jsr((num_smallRyRneighbors+1)*NSFU), PINNED=plog)    

ALLOCATE(Ca_ds_dev(0:(num_smallRyRneighbors+1)*NSFU), &
         Ca_jsr_dev(0:(num_smallRyRneighbors+1)*NSFU), stat=iserror)
\end{verbatim}
With new ``small RyR clusters'', we need to have more fluxes.
\begin{verbatim}
    ALLOCATE(J_efflux_dev(0:(num_smallRyRneighbors+1)*NSFU), &
         J_refill_dev(0:(num_smallRyRneighbors+1)*NSFU), &
         J_ryr_dev(0:(num_smallRyRneighbors+1)*NSFU), stat=iserror)
    
    !!NOTE: I_dhpr_dev not occur in small size RyR cluster
    ALLOCATE(I_dhpr_dev(0:NSFU), stat=iserror)
\end{verbatim}

  \item variable to keep the indices of current states of different CRU
  and rogue RyR clusters is increased in size
  \begin{verbatim}
ALLOCATE(sfu_state((num_smallRyRneighbors+1)*NSFU))
ALLOCATE(RYRgate((num_smallRyRneighbors+1)*NSFU),  DHPRgate(NSFU))

ALLOCATE(RYRgate_dev(0:(num_smallRyRneighbors+1)*NSFU),  &
        stat=iserror)
    
ALLOCATE(sfu_state_dev((num_smallRyRneighbors+1)*NSFU), STAT = iserror)
  \end{verbatim}
  
As we need to manage the states of rogue RyR clusters as well, we need to
modify the subroutine \verb!init_SFU_state!, i.e. replaced by
\verb!init_SFU_state_2!. 

\begin{framed}
SFU is replaced by CRU, i.e. \verb!init_CRU_state_2!. 
\end{framed}
%\verb!init_SFU_state_with_smallRyR!. 

%   \begin{verbatim}
%   ! called from loaddata_from_config()
% CALL init_SFU_state_with_smallRyR(N_L, mL, NUM_STATES_SFU, U_Ro, irow_R,  &
%          irow_R_small, U_Ro_small, sfu_state)
%   \end{verbatim}
  \begin{verbatim}
CALL init_SFU_state_2(N_L, mL, NUM_STATES_SFU, U_Ro, irow_R,  &
         irow_R_small, U_Ro_small, NSFU, sfu_state)
  \end{verbatim}

and
\begin{verbatim}
RYRgate_dev(1:NSFU) = U_Ro(sfu_state(1:NSFU))
RYRgate_dev(NSFU+1:) = U_Ro_small(sfu_state(NSFU+1:))
\end{verbatim}
\end{enumerate}

\subsection{Ver 2}
\label{sec:ver_2}

If we assume the peripheral RyR clusters see the bulk myoplasm and network SR,
then we need
\begin{enumerate}
  \item With new ``small RyR clusters'', we need to have more fluxes.
\begin{verbatim}
    ALLOCATE(J_ryr_dev(0:(num_smallRyRneighbors+1)*NSFU), stat=iserror)
    
    !!NOTE: I_dhpr_dev not occur in small size RyR cluster
    ALLOCATE(I_dhpr_dev(0:NSFU), stat=iserror)
\end{verbatim}

  \item variable to keep the indices of current states of different CRU
  and rogue RyR clusters is increased in size
  \begin{verbatim}
ALLOCATE(sfu_state((num_smallRyRneighbors+1)*NSFU))
ALLOCATE(RYRgate((num_smallRyRneighbors+1)*NSFU),  DHPRgate(NSFU))

ALLOCATE(RYRgate_dev(0:(num_smallRyRneighbors+1)*NSFU),  &
        stat=iserror)
    
ALLOCATE(sfu_state_dev((num_smallRyRneighbors+1)*NSFU), STAT = iserror)
  \end{verbatim}
  
As we need to manage the states of rogue RyR clusters as well, we need to
modify the subroutine \verb!init_SFU_state!, i.e. replaced by
\verb!init_SFU_state_with_smallRyR!. 

  \begin{verbatim}
  ! called from loaddata_from_config()
CALL init_SFU_state_with_smallRyR(N_L, mL, NUM_STATES_SFU, U_Ro, irow_R,  &
         irow_R_small, U_Ro_small, sfu_state)
  \end{verbatim}
and
\begin{verbatim}
RYRgate_dev(1:NSFU) = U_Ro(sfu_state(1:NSFU))
RYRgate_dev(NSFU+1:) = U_Ro_small(sfu_state(NSFU+1:))
\end{verbatim}
\end{enumerate}

\section{Data don't change}

\begin{enumerate}
  \item As we only use the normal release site to determine the time step, we
  don't change \verb!compPdt_dev! size
\begin{verbatim}
	! new method is used by default
    ALLOCATE(compPdt_dev(NSFU,1), stat=iserror)  
\end{verbatim}  

\item The vetor that tell how many DHPR channel opens at each CRU
\begin{verbatim}
ALLOCATE(DHPRgate(NSFU))

DHPRgate_dev = U_Lo(sfu_state(1:NSFU))

\end{verbatim}

\end{enumerate}

\section{Time step}

The adaptive time step is chosen from the normal CRUs, not from the rogue RyR.
So, we only need
\begin{verbatim}
ALLOCATE(compPdt_dev(NSFU,1), stat=iserror)  
\end{verbatim}
rather than
\begin{verbatim}
ALLOCATE(compPdt_dev((num_smallRyRneighbors+1)*NSFU,1), stat=iserror)  
\end{verbatim}

Also, as using the smaller grid point (0.1$\mum^3$,rather than 0.2$\mum^3$), we
need to reduce the time step upper bound from 1d-6 to 2d-7. 

\section{Random numbers}

However, as we simualte stochastically both regular and small size RyR
clusters, we need to change the amount of PRNs to be generated
\begin{verbatim}    
    !X_r_dev = keep pseudo-random numbers
    ALLOCATE(X_r_dev((num_smallRyRneighbors+1)*NSFU*copy_interval)) 
\end{verbatim}
This requires us to use a different function: \verb!init_PRNG_2()!, rather than
\verb!init_PRNG()!, with
\begin{verbatim}
    saruObj = init_saru(NSFU*(num_smallRyRneighbors+1), global_seed)
!!and
    ALLOCATE(X_r((num_smallRyRneighbors+1)*NSFU*copy_interval))
\end{verbatim}

\begin{framed}
NOTE: New code from branch 2 version, use only one function \verb!init_PRNG()!
that work for both regular w/wo rogue cluster of RyRs. 
\end{framed}

\section{Putting rogue RyRs}

UPDATE:
\begin{enumerate}
  \item \verb!_ADD_SMALL_RYRS_! macro is obsolete
  \item Rogue RyRs is replaced for small RyR cluster
  \item Functions to put rougue cluster is explicitly defined in different
  names, allowing us to implement a function to distribute rogue in any place
  easily
  \begin{verbatim}
  put_extra_1_RYRs_cluster
  put_extra_2_RYRs_cluster
  put_extra_3_RYRs_cluster
  put_extra_6_RYRs_cluster
  \end{verbatim}
\end{enumerate}

With rogue RyR, how to know the location of them in the grid? There are
two ways
\begin{enumerate}
  \item if the distance from each rogue RyR cluster to the master CRU is the
  same, we can consider each rogue RyR cluster is a ``pseudo'' CRU, so we
  increase the array to manage location of CRUs in the grid from 1..NSFU to 
  1..(num\_neighborsRyR+1)*NSFU.
  \begin{verbatim}
  allocate(SFU_in_lingrid( (num_smallRyRneighbors+1)*NSFU))
  \end{verbatim}
  Then, we manage them using the information of the CRU and the
  \verb!stride2smallRyR! plus the \verb!num_smallRyRneighbors! (initiated
  inside \verb!init_Simulation()!). For example: the $i$-th CRU has its
  upper neighbor at index (i+NSFU)-th index, left neighbor at (i+2*NSFU)-th
  index\ldots
\begin{verbatim}
!!put_SFU2grid3D_homo_small_RYR()
IF (dist2smallRyR .GT. 0.0d0) THEN 
   !put extra CRU here
            #if _ADD_SMALL_RYRS_ == _1_SMALL_RYRS
   CALL put_extra_1_RYRs_cluster(idx, ii, jj, kk)
            #elif _ADD_SMALL_RYRS_ == _3_SMALL_RYRS
   CALL put_extra_3_RYRs_cluster(idx, ii, jj, kk)
            #elif _ADD_SMALL_RYRS_ == _6_SMALL_RYRS
   CALL put_extra_6_RYRs_cluster(idx, ii, jj, kk)
            #endif

\end{verbatim}  
example:
\begin{verbatim}
!+ one on up
SFU_in_grid3D(idx+NSFU, 1:3) = (/ ii-stride2smallRyR, jj, kk /)
grid3D_with_SFU(ii-stride2smallRyR, jj, kk) = idx+NSFU

lingrid_with_SFU = RESHAPE(grid3D_with_SFU, (/ SIZEWITHBOUNDARY /) )
\end{verbatim}
and modify
\begin{verbatim}
tscell_layout_CaRUs2grid_linear()
\end{verbatim}
to use \verb!put_SFU2grid3D_homo_small_RYR!.

  \item if the distance can be different (using some radomly distribution), we
  need to split the handling of rogue RyR in two separate vectors/matrices:
  \verb!rogueRYR_in_lingrid! and \verb!lingrid_with_rogueRyR!. 
\begin{verbatim}
!data_tscell.f95
INTEGER, DIMENSION(:), ALLOCATABLE :: rogueRyR_in_lingrid
INTEGER, DIMENSION(:), ALLOCATABLE :: lingrid_with_rogueRyR
  !or
INTEGER, DIMENSION(:,:,:), ALLOCATABLE :: grid3D_with_rogueRyR  
\end{verbatim}
by modifying \verb!tscell_allocate_linear()! and
\verb!tscell_allocate_gpu_linear! (to allocate). Here, the size of
\verb!SFU_in_lingrid! doesn't change.

To deal with these two new arrays, we use
\begin{verbatim}
tscell_layout_CaRUsRogueRyR2grid_linear(SFU_in_lingrid, lingrid_with_SFU,
        rogueRyR_in_lingrid, lingrid_with_rogueRyR, put_SFUrogueRyR2grid3D_homo)
\end{verbatim}

\end{enumerate}

\section{Modified kernels}
\label{sec:kernel_change_peripheralRyR}

To set-up kernel configuration, we use \verb!setup_KernelConfiguration_2()!,
rather than \verb!setup_KernelConfiguration()!.
\begin{verbatim}
    CALL setup_KernelConfiguration_2()
\end{verbatim}

\begin{framed}
From branch 2 version, the code use only one name
\verb!setup_KernelConfiguration()! that works for all cases. 
\end{framed}

For kernel to update CRU state, we need to write a new kernel to update
the rogue RyR clusters. There are two options: use the single kernel to update
both CRU and rogue RyRs clusters. However, the code can be divergent. So, we can write two separate
kernels, one to update CRU and one to update rogue RyRs both run in parallel. 
\begin{itemize}
  \item Option 1: Use a single kernel to update both. This kernel uses
\begin{verbatim}
gridSize_smallRyR = CEILING(REAL((num_smallRyRneighbors+1)*NSFU)/blocksize)  
\end{verbatim}
and the kernel to update CRU/rogue-RyR states is
\verb!update_SFU_lin_with_smallRyR!.

\item Option 2: Update CRU and rogue RyRs in two separate kernels. The first one
to update CRUs use
\begin{verbatim}
gridSize = CEILING(REAL(NSFU)/blocksize)
\end{verbatim}
and the second one uses
\begin{verbatim}
gridSize_smallRyR = CEILING(REAL(num_smallRyRneighbors*NSFU)/blocksize)
\end{verbatim}
with the kernel to update CRU is \verb!update_SFU_lin!, and kernel to update
rogue-RyR is \verb!update_rogueRyR!. 
\end{itemize}

To select which approach, they can be reconfigured using macro
\begin{verbatim}
#define _UPDATE_SEPARATE 0
#define _UPDATE_BOTH     1
#define _APPROACH_2_UPDATE_CRU_ROGUERYR _UPDATE_SEPARATE
\end{verbatim}

\subsection{Update states (CRUs + rogue RyRs)}

\subsubsection{First approach}

We use a single kernel to update states of CRUs + rogue RyR clusters
\begin{verbatim}
!! update CRU + rogue RyR clusters in the same kernel
CALL update_SFU_lin_with_smallRyR <<< gridSize_smallRyR, blocksize>>> &
     (X_r_dev((iinner-1)*(num_smallRyRneighbors+1)*NSFU+1:(iinner)*(num_smallRyRneighbors+1)*NSFU), &
     dt, invdt, Vm, kva, kvi, dp_arg1, dp_arg2, &
     Ca_myo_1_dev(PADDING_D+1:), Ca_nsr_1_dev(PADDING_D+1:), &
     time, num_smallRyRneighbors, stride2smallRyR, &
     maxnkRp_small, maxnkRm_small &
     )
\end{verbatim}

\subsubsection{Second approach}

We use two separate kernels: one to update states of CRUs, and one to update
states of rogue RyR clusters.

\begin{verbatim}
CALL update_SFU_lin <<<gridSize, blocksize>>> &
     (X_r_dev((iinner-1)*(num_smallRyRneighbors+1)*NSFU+1:(iinner-1)*(num_smallRyRneighbors+1)*NSFU+NSFU), &
     dt, invdt, Vm, kva, kvi, dp_arg1, dp_arg2, &
     Ca_myo_1_dev(PADDING_D+1:), Ca_nsr_1_dev(PADDING_D+1:), &
     time &
     )
CALL update_rogueRyR <<< gridSize_smallRyR, blocksize>>> &
     (X_r_dev((iinner-1)*(num_smallRyRneighbors+1)*NSFU+NSFU+1:(iinner)*(num_smallRyRneighbors+1)*NSFU), &
     dt, invdt, Vm, kva, kvi, dp_arg1, dp_arg2, &
     Ca_myo_1_dev(PADDING_D+1:), Ca_nsr_1_dev(PADDING_D+1:), &
     time, num_smallRyRneighbors, stride2smallRyR, &
     maxnkRp_small, maxnkRm_small &
     )
\end{verbatim}

\subsection{update\_rogueRYR\_nochangestate}


\verb!update_rogueRYR_nochangestate! accepts \verb!num_rogueRYRneighbors!. It
assumes all CRUs have the same number of rogue-neighbors. 

The assumption above is relaxed in \verb!update_rogueRYR_nochangestate_3!.

\subsection{update\_rogueRYR\_nochangestate\_3}

This version, \verb!update_rogueRYR_nochangestate_3!, accepts
\verb!num_rogueRYRcluster! which is the number of total rogueRYRclusters.

\verb!update_rogueRYR_nochangestate! is the special case of this version where
\begin{verbatim}
num_rogueRYRcluster = num_rogueRYRneighbors * NSFU
\end{verbatim}

\subsection{update\_rogueRYR\_nochangestate\_4}

This version, \verb!update_rogueRYR_nochangestate_4! is an improvement to
\verb!update_rogueRYR_nochangestate_3!, as it allows rogue-RYRcluster to spread
more than one grid-point. User have to update the code and modify the macro
\verb!_ROGUE_LAYOUT_STRATEGY_!.
 
\begin{verbatim}
   #if _ROGUE_LAYOUT_STRATEGY_ == _2x2x1
       INCLUDE "2x2x1_Cansr.inc"
       INCLUDE "2x2x1_Camyo.inc"
   #else
       s_Ca_myo(tx) = Ca_myo(offset)
       s_Ca_nsr(tx) = Ca_nsr(offset)
   #endif    
\end{verbatim}


\subsection{Update grid elements}

We use \verb!update_GridElement_with_rogueRyR!. 


\begin{framed}
IDEA: \textcolor{red}{We may need to convert from using device array to
registers for some variables to increase performance}, e.g. \verb!J_efflux_dev(:)! array
to \verb!J_efflux! variable, \verb!J_refill_dev()!. CHALLENGE: We need to use
them in \verb!ReactTerm_Cmyo! and \verb!ReactTerm_Cnsr!.
\end{framed}

For data in simulation with rogue RyRs, it's better to use the zero-th
index. So, the new formula is

Then, we need special settings
%     Ca_ds_dev(1:NSFU) = Ca_ds(1:NSFU)
%     Ca_jsr_dev(1:NSFU) = Ca_jsr(1:NSFU)
%     Ca_ds_dev(NSFU+1:) = Ca_myo
%     Ca_jsr_dev(NSFU+1:) = Ca_nsr
% 
\begin{verbatim}
    Ca_ds_dev(1:) = Ca_ds
    Ca_jsr_dev(1:) = Ca_jsr

    Ca_ds_dev(0) = 0.0d0
    Ca_jsr_dev(0) = 0.0d0
    J_efflux_dev(0) = 0.0d0
    J_refill_dev(0) = 0.0d0
    J_ryr_dev(0) = 0.0d0
    I_dhpr_dev(0) = 0.0d0
    RYRgate_dev(0) = 0.0d0
    DHPRgate_dev(0) = 0.0d0

    sfu_state_dev = sfu_state
    RYRgate_dev(1:NSFU) = U_Ro(sfu_state(1:NSFU))
    RYRgate_dev(NSFU+1:) = U_Ro_small(sfu_state(NSFU+1:))
    DHPRgate_dev = U_Lo(sfu_state(1:NSFU))
    U_Ro_dev = U_Ro
    U_Lo_dev = U_Lo
    U_Ro_small_dev = U_Ro_small

\end{verbatim}


% The compact-form rate-transition matrices for the small cluster is created by
% this code
% \begin{verbatim}
%     IF (has_smallRyR .EQ. 1) THEN
% #if _USE_TRANSPOSE_FORM_ == _YES
%        CALL makecompKs_RYRsmall_T(N_R_small, compK_Rm_small, compK_Rp_small, &
%             idx_Rm_small, idx_Rp_small, &
%             U_Ro_small, irow_R_small, maxnkRm_small, maxnkRp_small, setupRyR_2state_T)
%        
% #else
%        CALL makecompKs_RYRsmall(N_R_small, compK_Rm_small, compK_Rp_small, &
%             idx_Rm_small, idx_Rp_small, &
%             U_Ro_small, irow_R_small, maxnkRm_small, maxnkRp_small, setupRyR_2state)
%        
% #endif
%     ENDIF
% \end{verbatim}

\section{Code to analyze result}

\subsection{write\_peripheralRyR2file}

The function \verb!write_peripheralRyR2file! prints out data related to trigger
of rogue RyR. It uses
\begin{verbatim}
LOGICAL : is_written2peripheralRyR = .FALSE.
\end{verbatim}
A new variable is needed: \verb!flag5!
\begin{verbatim}
   ! flag5 (tell if CRU/rogueRyR is triggered)  
    ALLOCATE(flag1((num_smallRyRneighbors+1)*NSFU), flag2((num_smallRyRneighbors+1)*NSFU), &
         flag3((num_smallRyRneighbors+1)*NSFU), flag4((num_smallRyRneighbors+1)*NSFU), stat=iserror)
    IF (iserror) PRINT *, "MEM ERROR: flag1, flag2, flag3, flag4"
    ALLOCATE(flag5((num_smallRyRneighbors+1)*NSFU), stat=iserror)
    IF (iserror) PRINT *, "MEM ERROR: flag5"    
    flag1 = .FALSE.
    flag2 = .FALSE.
    flag3 = .FALSE.
    flag4 = .FALSE.
    flag5 = .FALSE.
    ALLOCATE(ts_spark((num_smallRyRneighbors+1)*NSFU), &
         ts_blink((num_smallRyRneighbors+1)*NSFU), stat=iserror)
    IF (iserror) PRINT *, "MEM ERROR: ts_spark, ts_blink"
    ts_spark = +0d0
    ts_blink = +0d0   
\end{verbatim}

The 5 new files are created. Each file print out the corresponding data at the
time all RYRs are opened at normal CRU (first), and then rogue RYRs
\begin{verbatim}
open_time.txt  //time 
open_cmyo.txt  //[Ca2+] at that location
open_CaF.txt   //[CaF] at that location
open_NoRYR.txt // #openRYR at that location
open_Cads.txt  // [Ca]ds at that location 
              // (in the case of atrial cell, it would be the same as
              // open_cmyo.txt)
\end{verbatim}  



\subsection{trigger\_rogue\_RYR}

\verb!trigger_rogue_RYR!, this version does some analysis by triggering the
rogue RyR clusters, not the main CRU.
\begin{verbatim}
  SUBROUTINE trigger_rogue_RYR(U_Ro_small, sfu_state, NSFU, num_smallRyRneighbors, is_all_rogueRyR_open)
\end{verbatim}
This need to uses some variables
\begin{verbatim}
INTEGER, DIMENSION(:), ALLOCATABLE ::  index_rogueRyR_numRyRopen  ! index where #openRyR is 1, 2, ... N_R_small
\end{verbatim}

NOTE: \verb!num_smallRyRneighbor! is replaced by \verb!num_rogueRYRcluster!.


\subsection{trigger\_rogue\_RYR\_3}

\verb!trigger_rogue_RYR_3!, this version relaxes the assumption that all CRUs
have the same number of rogue-RyR cluster. It accepts the total number of
rogue-RYR cluster \verb!num_rogueRYRcluster!.

In this version, all rogueRYRclusters are triggered.

\subsection{trigger\_rogue\_RYR\_4}

\verb!trigger_rogue_RYR_4! is the variant of \verb!trigger_rogue_RYR_3!, as it
allows user to trigger one particular cluster of rogueRYRclusters only. Here, it
assumes there are 3 rogueRYRclusters, and it triggers the second one. 

\section{File changes}
\label{sec:rogueRyR_filechanges}

\subsection{data\_tscell.f95}
\label{sec:data_tscell}

OBSOLETE: \verb!..._small...! becomes \verb!\..._rogue...!

\begin{verbatim}
  INTEGER :: stride2rogueRyR
  REAL(KIND=dp) :: dist2smallRyR
  INTEGER :: num_rogueRyRneighbors = 0, N_R_rogue
  REAL(KIND=dp) :: V_ds_rogue, lambda_ds_rogue, inv_lambda_ds_rogue
  REAL(KIND=dp) :: Ej_rogue
  !REAL(KIND=dp), DIMENSION(:,:), ALLOCATABLE :: ryr_KCal_act_rogue, ryr_Kfree_rogue
\end{verbatim}

New variables:
\begin{verbatim}
  INTEGER :: stride2smallRyR
  REAL(KIND=dp) :: dist2smallRyR
  INTEGER :: num_smallRyRneighbors = 0
  REAL(KIND=dp) :: V_ds_small, lambda_ds_small, inv_lambda_ds_small
\end{verbatim}
then
\begin{verbatim}
    !! prepare ...	
    IF (dist2smallRyR .GT. 0.0d0) THEN 
       !put extra CRU here
#if _ADD_SMALL_RYRS_ == _1_SMALL_RYRS
       ALLOCATE(SFU_in_grid3D(2*NSFU, 3))
       num_smallRyRneighbors = 1
#elif _ADD_SMALL_RYRS_ == _3_SMALL_RYRS
       ALLOCATE(SFU_in_grid3D(4*NSFU, 3))
       num_smallRyRneighbors = 3
#elif _ADD_SMALL_RYRS_ == _6_SMALL_RYRS
       ALLOCATE(SFU_in_grid3D(7*NSFU, 3))
       num_smallRyRneighbors = 6
#else
       WRITE(log_string, *) "Unsupported num neighboring small RyR cluster"
       call write2log(log_string)
       WRITE(log_string, *) "...please, update _ADD_SMALL_RYRS_"
       CALL write2log(log_string)
       STOP       
#endif
       CALL(log_string, *) "num_smallRyRneighbors = ", num_smallRyRneighbors
       CALL write2log(log_string)
    ELSE
#if _ADD_SMALL_RYRS_ != _0_SMALL_RYRS
       WRITE(log_string, *) "Wrong setting with small RyRs"
       CALL write2log(log_string)
       STOP
#endif
       ALLOCATE(SFU_in_grid3D(NSFU, 3))
    END IF
\end{verbatim}
then, a new subroutine is defined to put both normal CRU and rogure RyR clusters
in the mesh grid.
\begin{verbatim}
FUNCTION
  put_SFU2grid3D_homo_small_RYR (SFU_ingrid3D, grid3Dwith_SFU) RESULT (res)
\end{verbatim}


\subsection{tscell\_gpu\_utility.f95}
\label{sec:tscell_gpu_utility}

New variables: to keep track the rate transition of the small clusters of rogue
RyRs
\begin{verbatim}
  REAL(KIND=dp), DIMENSION(:,:), ALLOCATABLE, device :: compK_Rm_small_dev, compK_Rp_small_dev
  INTEGER, DIMENSION(:, :), ALLOCATABLE, device :: idx_Rm_small_dev, idx_Rp_small_dev
  REAL(KIND=dp), DIMENSION(:), ALLOCATABLE, device :: U_Ro_small_dev
\end{verbatim}
with data allocation
\begin{verbatim}
    ALLOCATE(compK_Rm_small_dev(0:maxnkRm_small, irow_R_small), &
         compK_Rp_small_dev(0:maxnkRp_small, irow_R_small), STAT=iserror)
    IF (iserror .NE. 0) PRINT *, "MEM ERROR: compK_Rm_small_dev..."
    ALLOCATE(idx_Rm_small_dev(0:maxnkRm_small, irow_R_small), &
         idx_Rp_small_dev(0:maxnkRp_small, irow_R_small) )
    IF (iserror .NE. 0) PRINT *, "MEM ERROR: index_Rm_small_dev, ..."
    
    ALLOCATE(U_Ro_small_dev(irow_R_small), stat=iserror)
    IF (iserror) PRINT *, "MEM ERROR: U_Ro_small_dev" 
\end{verbatim}
The gating for these small clusters are contained in the extended vector of
\verb!RYRgate_dev()! (NOTE: there is no change in \verb!DHPRgate_dev()!)
\begin{verbatim}
    ALLOCATE(RYRgate_dev(0:(num_smallRyRneighbors+1)*NSFU),  &
         DHPRgate_dev(0:NSFU), stat=iserror)
\end{verbatim}

The vector \verb!sfu_state_dev()! that keeps the current state index is
also extended. 
\begin{verbatim}
    ALLOCATE(sfu_state_dev((num_smallRyRneighbors+1)*NSFU), STAT = iserror)
\end{verbatim}
NOTE: from 1..NSFU, it keeps index for vector \verb!U_Ro!, while from
NSFU+1\ldots to the end, it keeps the index for vector \verb!U_Ro_small!. 

As the gating for channels in these small clusters are stochastic as well, we
also extend the number of generated pseudo-random numbers
\begin{verbatim}
!X_r_dev = keep pseudo-random numbers
    ALLOCATE(X_r_dev((num_smallRyRneighbors+1)*NSFU*copy_interval), stat=iserror) 
    IF (iserror) PRINT *, "MEM ERROR: X_r_dev"
\end{verbatim}
NOTE: We need to modify \verb!init_PRNG()! to \verb!init_PRNG_2()!.

% Also, we assume there is a small subspace area in these small cluster, so with
% $N$ such small clusters, we need to have $N$ new calcium subspace. To make it
% easier to handle, we extend the array \verb!Ca_ds_dev()! and \verb!Ca_jsr_dev()! to
% \begin{verbatim}
%     ALLOCATE(Ca_ds_dev(0:(num_smallRyRneighbors+1)*NSFU), &
%          Ca_jsr_dev(0:(num_smallRyRneighbors+1)*NSFU), stat=iserror)
%          
%     ALLOCATE(Ca_ds_prev_dev(0:(num_smallRyRneighbors+1)*NSFU), stat=iserror) 
% \end{verbatim}
% along with these subspaces, we also need addition fluxes variables, so we also
% extend the vectors of the fluxes
% \begin{verbatim}
%     ALLOCATE(J_efflux_dev(0:(num_smallRyRneighbors+1)*NSFU), &
%          J_refill_dev(0:(num_smallRyRneighbors+1)*NSFU), &
%          J_ryr_dev(0:(num_smallRyRneighbors+1)*NSFU), stat=iserror)
%     J_efflux_dev = +0d0; J_refill_dev = +0d0; J_ryr_dev = +0d0
% \end{verbatim}
% NOTE: The zero-th entry is important in device kernel for efficient calculation.
% In host code, we don't need to use this


% In kernel, we have some changes: we have another grid size to work with both
% normal and small clusters
% \begin{verbatim}
% blocksize = BLOCKSIZE_1D
% gridSize = CEILING(REAL(NSFU)/blocksize)
% gridSize_smallRyR = CEILING(REAL((num_smallRyRneighbors+1)*NSFU)/blocksize)  ! ceiling value
% \end{verbatim}

For other data changes, see Sect.\ref{sec:size_data_change}. For kernel changes,
see Sect.\ref{sec:kernel_change_peripheralRyR}.
 
\section{Name changes}

\begin{enumerate}
  \item We change \verb!_small! to \verb!_rogue!, e.g
  \begin{verbatim}
  INTEGER :: num_rogueRyRneighbors
  \end{verbatim}
  So, the new names are
  \begin{verbatim}
  lambda_ds_small     -->   lambda_ds_rogue
  inv_lambda_ds_small -->   inv_lambda_ds_rogue
  V_ds_small          -->   V_ds_rogue
  N_R_small           -->   N_R_rogue
  E_j_small           -->   E_j_rogue
  \end{verbatim}
  The transition matrices and indexing matrices as well
  \begin{verbatim}
  compK_Rm_small      -->   ryr_compKfree_rogue
  compK_Rp_small      -->   ryr_compKcal_act_rogue3
  \end{verbatim}
  
\end{enumerate}

