\chapter{EC-coupling}
\label{sec:ec-coupling}

\begin{verbatim}
irow_L      (number of LCC-cluster states)                 [unitless]

NUM_STATES_SFU  (number of CRU states)                     [unitless]
            (a heterogeneous cluster of LCC & RyR)
            = irow_L * irow_R

DHPRgate    (number of DHPRs open at each CRU)             [unitless]
            array of size NSFU


## gating variables  [fraction of channel in open or closed state]
m_Na                                                       [unitless]
           0.0015d0
h_Na
           0.9849d0
j_Na 
           0.9905d0
a_Ktof
           0.0021d0
in_Ktof
           1.0d0
a_Ktos
           2.9871d-4
in_Ktos  
           0.9994d0 
a_Kss
           0.002d0
in_Kss  
           1.0d0
           
## parameters for gating [transition rate from open to close and vice verse]
a_h, a_j, b_h, b_j, a_m, b_m                                        [1/sec]
#NOTE: Luo-Rudy use the unit of [1/ms], so here 
#      all the gatings need to multiply with 1d3. 

\end{verbatim}

\begin{verbatim}
sigma_v_dhpr  (activation parameter for DHPR Vm-dependent)          (mV)
              = 6.24d0 mV
Vtheta_dhpr   (activation threshold)                                (mV)
              5d0
# Vm-dependent transition rates on 2 different states
# (mode normal, and mode calcium) of 6-state LCC 
#  S1 -> S2 -> S3 (S3=open)
#  |     |     |
#  S4 -> S5 -> S6
kva     (S1->S2)                                        [unitless]
        = dexp((Vm - Vtheta_dhpr)/sigma_v_dhpr)/(1d0 + dexp((Vm - Vtheta_dhpr)/sigma_v_dhpr))
kvi     (S4->S5)                                        [unitless]
        = 1d0 - (1d0/(1d0+EXP((Vm+21.5d0)/4.5d0)) + 0.27d0/(1d0 + EXP((30d0-Vm)/8d0)))
\end{verbatim}
\begin{equation}
  \label{eq:12}
  \begin{split}
  kva &=
  \frac{\exp(\frac{Vm-V^\theta_\dhpr}{\sigma_{v,\dhpr}})}{1+\exp(\frac{Vm-V^\theta_\dhpr}{\sigma_{v,\dhpr}})} \\
   kvi &= 1- \left( \frac{1}{1+\exp(\frac{Vm+21.5}{4.5})} + \frac{0.27}{1+\exp(\frac{30-Vm}{8})}\right)
  \end{split}
\end{equation}
with $V^\theta_\dhpr=5.0$ mV, $\sigma_{v,\dhpr}=6.24$ mV.

\begin{verbatim}
## NOTE: 1 = RyR, 2 = LCC
# Ca2+-dependent transition S2 -> S3 of LCC
bigC1(ii) = Ca_ds(ii)**eta_LCC                              [uM^eta_LCC]

# Ca2+-dependent transition C->O of RyR
bigC2(ii) = Ca_ds(ii)**eta_RyR                              [uM^eta_RyR]
# allosteric coupling transition O->C of RyR
chiC(ii)                                                   [unitless]
           = EXP(-Ej*0.5d0*((N_R-RYRgate(ii))*Ecc - (RYRgate(ii)-1d0)*Eoo))
# allosteric coupling transition C->O of RyR
chiO(ii)                                                   [unitless]
           = EXP(-Ej*0.5d0*(RYRgate(ii)*Eoo - (N_R-RYRgate(ii)- 1d0)*Ecc))

# luminal Ca2+-dependent gating C->O of RyR
phi(ii)                                                    [????]
           = k_jsr0*Ca_jsr(ii) + k_jsr1
\end{verbatim}
See eq.~\eqref{eq:24}.

\section{Reversal potential}
\label{sec:reversal-potential-1}

\begin{verbatim}
          E_Na = RT_F*LOG(Na_o/Na_i)
          E_K = RT_F*LOG(K_o/K_i)
          E_Ca = RT_FzCa*LOG(Ca_o/Ca_myo)  
\end{verbatim}
\begin{equation}
  \label{eq:31}
  \begin{split}
      E_\na &=  \frac{RT}{z_\na F}\log\frac{[\Na]_o}{[\Na]_\myo} \\
      E_\k &=  \frac{RT}{z_\k F}\log\frac{[\K]_o}{[\K]_\myo}\\
      E_\ca &=  \frac{RT}{z_\ca F}\log\frac{[\Ca]_o}{[\Ca]_\myo}
  \end{split}
\end{equation}

\section{Gating}
\label{sec:gating}

\textcolor{red}{The gatings units are [1/sec]}. 

\begin{verbatim}
# Using the heuristic formula of Luo-Rudy for Na current and K current
!! Fast Na current (Luo-Rudy)
IF (Vm .GE. -40.d0) THEN
   a_h = 1d3*0
   a_j = 1d3*0
   b_h = 1d3*1d0/(0.13d0*(1d0+EXP((Vm+10.66d0)/-11.1d0)))
   b_j = 1d3*0.3d0*(EXP(-2.535d-7*Vm)/(1+EXP(-0.1d0*(Vm+32d0))))
ELSEIF (Vm .LT. -40.d0) THEN 
   a_h = 1d3*(0.135d0*EXP((80d0+Vm)/-6.8d0))
   a_j = 1d3*(-1.2714d5*EXP(0.2444d0*Vm)-3.474d-5*EXP(-0.04391d0*Vm))* &
        ( (Vm+37.78d0)/(1+EXP(0.311d0*(Vm+79.23d0))))
   b_h = 1d3*(3.56d0*EXP(0.079d0*Vm)+3.1d5*EXP(0.35d0*Vm))
   b_j = 1d3*(0.1212d0*EXP(-0.01052d0*Vm)/(1+EXP(-0.1378d0*(Vm+40.14d0))))
ELSE ! NaN blank
   PRINT *, Vm
   PRINT *, "Error: NaN value Vm"
   RETURN
END IF  

a_m = 1d3*0.32d0*(Vm+47.13d0)/(1d0-EXP(-0.1d0*(Vm+47.13d0)))
b_m = 1d3*0.08d0*EXP(-Vm/11d0)
\end{verbatim}
\begin{equation}
  \label{eq:33}
  \begin{split}
      a_m &= 10^3\times 0.32 \times
      \frac{V_m+47.13}{1-\exp(-0.1(V_m+47.13))}\\
      b_m &= 10^3\times 0.08 \times \exp(-\frac{V_m}{11})
  \end{split}
\end{equation}

\textcolor{red}{The time must be used in [sec], otherwise unexpected
  results can occur}.
The fraction of gating particles on one side of the membrane are given
below
\begin{verbatim}
m_Na = m_Na + dt*(a_m*(1-m_na)-b_m*m_na)           [unitless]
h_Na = h_Na + dt*(a_h*(1-h_na)-b_h*h_na)
j_Na = j_Na + dt*(a_j*(1-j_na)-b_j*j_na)
a_Ktof = a_Ktof +
dt*(1d3*((0.18064d0*EXP(0.03577d0*(Vm+30d0)))*(1d0-a_Ktof) -
            (0.3956d0*EXP(-0.06237d0*(Vm+30d0)))*a_Ktof)) 

in_Ktof = in_Ktof + dt*(1d3*((((0.000152d0*EXP(-(Vm+13.5d0)/7d0))/
           (0.067083d0*EXP(-(Vm+33.5d0)/7d0)+1d0)))*
               (1-in_Ktof)-(((0.00095d0*EXP((Vm+33.5d0)/7d0))/
                   (0.051335d0*EXP((Vm+33.5d0)/7d0)+1d0)))*in_Ktof))   

a_Ktos = a_Ktos +
            dt*(1d3*((1d0/(1d0+EXP(-(Vm+22.5d0)/7.7d0)))-a_Ktos)/
                 (0.493d0*EXP(-0.0629d0*Vm)+2.058d0)) 
in_Ktos = in_Ktos + dt*(1d3*((1d0/(1+EXP( (Vm+45.2d0)/5.7d0))) -
                 in_Ktos) /((270.0d0+1050d0/(1+EXP( (Vm+45.2d0)/5.7d0 )))))
a_Kss  = a_Kss  + dt*(1d3*((1d0/(1d0+EXP(-(Vm+22.5d0)/7.7d0)))-a_Kss)/
                      (0.393d0*EXP(-0.0862d0*Vm)+0.13d0)) 
in_Kss  = in_Kss  + dt*0  
\end{verbatim}

With the non-junctional DHPR and RyR, we use deterministic approach
\begin{verbatim}
tmp_pL_nj = pL_nj
DO ii=1, irow_L_nj
   pL_nj(ii) = tmp_pL_nj(ii) + dt*SUM(tmp_pL_nj(:)*(kva*aKv1_nj(:,ii)+ 
               kvi*aKv2_nj(:,ii) + aKm1_nj(:,ii) + &
               Ca_myo**eta_LCC * aKp1_nj(:,ii)))
END DO
tmp_pR_nj = pR_nj
DO ii=1, irow_R_nj
   pR_nj(ii) = tmp_pR_nj(ii) + dt*SUM(tmp_pR_nj(:)*(aKm2_nj(:,ii) +
               (k_jsr0*Ca_nsr+k_jsr1)*Ca_myo**eta_RyR * aKp2_nj(:,ii))) 
ENDDO
\end{verbatim}

\section{Membrane ionic currents}
\label{sec:membr-ionic-curr-1}

\begin{framed}
  NOTE: as $\overline{g}_i$ is in mS/cm$^2$, and $V_m,E_i$ in mV, then
  the resulting current is in $\mu$A/cm$^2$.
\end{framed}
\begin{verbatim}
# Fast Na current [uA/cm^2]
I_Na = g_na * m_na**3 * h_na * j_na * (Vm - E_Na)
\end{verbatim}
\begin{equation}
  \label{eq:32}
  I_\na = g_\na \times m_\na^3 \times h_\na \times j_\na (V_m-E_\na)
\end{equation}

\begin{verbatim}
 ## K currents
 I_Ktof = g_Ktof* a_Ktof**3 *in_Ktof*(Vm-E_K)
 I_Ktos = g_Ktos* a_Ktos**3 *in_Ktos*(Vm-E_K)
 I_K1   = g_k1*(K_o/(K_o+210.0d0))*((Vm-E_K-1.73d0)/
                 (1+EXP(0.0596d0*(Vm-E_K-1.73d0)))) 
 I_Kss  = g_Kss*a_Kss*in_Kss*(Vm-E_K) 
\end{verbatim}

\begin{verbatim}
 ## Background currents
 I_bNa = g_bNa*(Vm-E_Na)
 I_bK  = g_bK*(Vm-E_K)
 I_bCa = g_bCa*(Vm-E_Ca) 
\end{verbatim}


\begin{verbatim}
## Sarcolemma pumps and Exchangers
I_NaK  = I_NaK_bar*(1d0/(1d0+0.1245d0*EXP(-0.1d0*(Vm*F_RT))+
             0.0365d0*(1/7d0*(EXP(Na_o/67300d0)-1d0))* & 
             EXP(-(Vm*F_RT))))*(1d0/(1d0+(21000d0/Na_i)**1.5d0))*
             (K_o/(K_o+1500d0))  

I_pmca = I_pmca_bar*(Ca_myo**eta_pmca/(Km_pmca**eta_pmca+Ca_myo**eta_pmca))

I_ncx  = I_ncx_bar*(Na_i**3 * Ca_o*EXP(eta_ncx*Vm*F_RT) - Na_o**3 *
             Ca_myo*EXP((eta_ncx-1d0)*Vm*F_RT))/ & 
               ((Km_ncxNa**3 + Na_o**3)*(Km_ncxCa+Ca_o)*
                 (1d0+ksat_ncx*EXP((eta_ncx-1d0)*Vm*F_RT))) 
\end{verbatim}
\begin{equation}
  \label{eq:39}
  \begin{split}
      I_\NaK = \overline{I}_\NaK \frac{1}{1+0.1245\exp(-0.1\frac{V_mF}{RT})) +
    (\frac{0.0365}{7\exp(\frac{[\Na]_o}{67300})-1}) \times \exp(-\frac{V_mF}{RT}) }
  \\
  \frac{1}{1+\left( \frac{21000}{[\Na]_i}\right)^{1.5}}\frac{[\K]_o}{[\K]_o+1500}
  \end{split}
\end{equation}
and eq.~\eqref{eq:10} ($I_\pmca$), eq.~\eqref{eq:11} ($I_\ncx$).

\textcolor{red}{The $\Ca$ current via DHPR is measured at whole-cell
  level, so the unit is $\mu$A}, rather than $\mu$A/cm$^2$. So, when we
  combine with other ionic currents to calculate the membrane
  potential, we need to divided it by the surface area of the membrane
  $A_m$.

\begin{verbatim}
 I_dhpr_nj_T                                                   [uA]
           = SUM(pL_nj, mask=U_Lo_nj .EQ. 1) * P_dhpr_nj*F2zCa2_RT*Vm * &
          (Ca_myo * EXP(Vm*FzCa_RT) - 0.341d0 * Ca_o)/(EXP(Vm*FzCa_RT)-1d0)  

## I_dhpr_nj_T is the current through all non-junctional DHPR (i.e. at
#  whole-cell level), so the unit of I_dhpr_nj_T is [uA]
#  Correspondingly, the unit of P_dhpr_nj is [L/sec]
P_dhpr_nj = P_dhpr_T * prcent_dhpr_nj                          [L/sec]
\end{verbatim}
\begin{equation}
  \label{eq:30}
  I_{\dhpr,\nj}^T = \sum
  \begin{array}{l}
    \text{fraction of channels}\\
    \text{ in open states}
  \end{array}
  P_{\dhpr,\nj}.\frac{z_\ca^2.F^2}{RT}.V_m.\frac{[\Ca]_\myo*\exp(\frac{z_\ca.V_m.F}{RT})
    - 0.341[\Ca]_o}{\exp(\frac{z_\ca.V_m.F}{RT})-1}
\end{equation}

\textcolor{red}{NOTE: Currently, we don't have $I_{\ca(T)}$ T-type
  current, non-specific $\Ca$ current $I_\nsCa$} in the model. T-type
channel inactivates at very negative membrane potential $V_m<-50$mV.

\subsection{Trans-membrane currents to each CRU}
\label{sec:inside-each-cru}

\begin{verbatim}
## total current via DHPR at each CRU [uA]
I_dhpr(ii)                                              [uA]
            = DHPRgate(ii) * P_dhpr * (F2zCa2_RT*Vm) * &
            (Ca_ds(ii) * EXP(Vm*FzCa_RT) - 0.341d0 * Ca_o) /(EXP(Vm*FzCa_RT) - 1)  

P_dhpr      (permeability for single LCC channel)       [10^-6 L/sec]
            = P_dhpr_T/(NSFU * N_L)
# NOTE: If V_myo_T is reduced half, we also need to reduce CRUs half
#       for these rates to be unchanged.
# However, in non-spatial simulation, we often run with a smaller
# amount of CRUs, while keeping the V_myo_T unchanged. That's why 
# we need to increase the permeability
# P_dhpr = P_dhpr_T / (NSFU * N_L)
\end{verbatim}
\begin{equation}
  \label{eq:34}
  I_\dhpr^\ii = N^\open_\dhpr .P_\dhpr.z_\ca^2.\frac{F^2 V_m}{RT}
  \frac{0.341[\Ca]_o-[\Ca]_\ds\exp(\frac{z_\ca
      V_mF}{RT})}{1-\exp(\frac{z_\ca V_mF}{RT})}
\end{equation}

\begin{framed}
  \begin{verbatim}
NOTE: zCa*Vm*F/RT is frequently used in many formula. We 
can avoid recalculating by defining
    zCaVmF_RT = zCa*Vm*F/RT
 \end{verbatim}
\end{framed}

NOTE: We may need to add \verb!I_ncx_j! as well. 


\section{Fluxes}
\label{sec:fluxes-1}


In transport phenomena (heat transfer, mass transfer), {\bf flux} is defined as
the rate of flow of a property (e.g. heat, ions) per unit area, which has the
dimension: [quantity].[time]$^{-1}$.[area]$^{-1}$. The flux is the 
\begin{equation}
j = \frac{dI}{dA}
\end{equation}
with $I$ is the current, and $A$ is the surface area. 

In diffusion flux, based on Fickian diffusion, the dimension of flux is :
(mol�m$^{-2}$�s$^{-1}$). With chemical diffusion, the [quantity] is in Molar,
e.g. $\muM$ or mM. The flux of flow from component A to component B
\begin{equation}
j = -D_{AB} \Delta c_A
\end{equation}
with $D_{AB}$ is diffusion constant



\textcolor{red}{All fluxes are defined over $V_\myo^T$}.

\subsection{Trans-sarcolemmal fluxes go to bulk myoplasm}
\label{sec:trans-sarc-flux-1}


\begin{verbatim}
# The current I_dhpr_nj_T is already measured at whole-cell level
# So, the unit is [uA]
# --> no need to multiply Am to find the flux J_dhpr_nj_T

J_Ca_sarmem = (-I_bCa - I_pmca + zCa * I_ncx) * Am_2FVmyo  -
                     I_dhpr_nj_T * inv_2FVmyo 
\end{verbatim}
\begin{equation}
  \label{eq:37}
  J_{\ca,sarmem} = (-I_\bCa - I_\pmca + z_\ca I_\ncx) \frac{A_m}{z_\ca
    FV_\myo^T} - I_{\dhpr,\nj}^T\frac{1}{z_\ca FV_\myo^T}
\end{equation}

\subsection{Trans-sarcolemmal fluxes go to CRU}
\label{sec:trans-sarc-flux-2}

\begin{verbatim}
# The current is already measured at whole-cell level
# So, the unit is [uA]
# --> no need to multiply surface area 
J_dhpr(ii) = -I_dhpr(ii) * inv_2FVmyo
\end{verbatim}

NOTE: We need to add \verb!J_ncx_j! if we have \verb!I_ncx_j!.

\subsection{Internal fluxes}
\label{sec:internal-fluxes-1}

Sect.~\ref{sec:internal-fluxes}.

\section{$\Ca$, $\K$, $\Na$ dynamics}
\label{sec:ca-k-na}

\begin{verbatim}
 Na_i = Na_i + dt*(-(I_Na + I_bNa + 3 *I_ncx + 3 *I_NaK)*Am_FVmyo)
 K_i = K_i + dt*(-(I_K1 + I_Kss + I_Ktof + I_Ktos + I_bK - 2*I_NaK)*Am_FVmyo)  
\end{verbatim}


The change of calcium in each CRU
\begin{verbatim}
## Ca_ds (use either of the 2 formula, the exact form requires dt .LE. 1d-7
# exact formula
Ca_ds(ii) = Ca_ds(ii) + dt*beta_ds/lambda_ds * (J_ryr(ii) -
                           J_efflux(ii) + J_dhpr(ii)) 
# approximation
dummy1 = DHPRgate(ii) * P_dhpr * F2zCa2_RT * Vm / (EXP(Vm*FzCa_RT)-1) * inv_2FVmyo
dummy2 = RYRgate(ii) * v_ryr
Ca_ds(ii) = (dummy2 * Ca_jsr(ii) + v_efflux * Ca_myo + dummy1 *
                  0.341d0 * Ca_o) / (dummy2 + v_efflux +dummy1)  

## Ca_jsr             
Ca_jsr(ii) = Ca_jsr(ii) + dt*(beta_jsr/lambda_jsr) * (J_refill(ii) - J_ryr(ii))  
\end{verbatim}
\begin{equation}
  \label{eq:28}
  \left[ 
    \begin{array}{l}
      \frac{d[\Ca]_\ds^\ii}{dt} = \frac{\beta_\ds}{\lambda_\ds}(J_\ryr^\ii -
      J_\ef^\ii + J_\dhpr^\ii)   \\
      \text{[}\Ca]_\ds^\ii =  \frac{f_1 [\Ca]_\jsr^\ii + v_\ef
        [\Ca]_\myo + 0.341 f_2 [\Ca]_o}{f_2 + v_\ef + f_1}
    \end{array}
\right.
\end{equation}
with 
\begin{equation}
  \label{eq:38}
  \begin{split}
    f_1 = N_\dhpr^{\open^\ii} P_\dhpr z_\ca^2F^2/(RT)\frac{V_m}{z_\ca
      FV_\myo^T(\exp(z_\ca FV_m/(RT)))} \\
    f_2 = N_\ryr^{\open^\ii} .v_\ryr
  \end{split}
\end{equation}
and
\begin{equation}
  \label{eq:29}
  \frac{d[\Ca]_\jsr^\ii}{dt} = \frac{\beta_\jsr}{\lambda_\jsr}(J_\rf^\ii-J_\ryr^\ii)
\end{equation}

The change of global calcium 
\begin{Verbatim}
Ca_nsr = Ca_nsr + dt*((beta_nsr/lambda_nsr)  * (J_serca - J_leak -
                        SUM(J_refill) - J_ryr_nj_T)) 
Ca_myo = Ca_myo + dt*(beta_myo * (SUM(J_efflux(:)) + J_leak +
                      J_Ca_sarmem + J_ryr_nj_T - J_serca - J_trpn)) 
CaB_htrpn = CaB_htrpn + dt*J_trpn
\end{Verbatim}

\section{$I_\stim$ clamp}
\label{sec:i_stim-clamp}

\begin{verbatim}
 ##I-clamp
 Vm = Vm + dt*(-1.0d3/Cm * (I_Na + (SUM(I_dhpr) + I_dhpr_nj_T)/Am + I_k1 + &
                I_Kss + I_Ktof + I_Ktos + I_ncx  + I_NaK + I_pmca + I_bNa +
                I_bCa + I_bK + I_app))   
\end{verbatim}
with 
\begin{verbatim}
I_app = I_stim * 1d-3 /Am
\end{verbatim}
with \verb!I_stim! in nA, and Am is the membrane area (cm$^2$).

\section{Voltage clamp}
\label{sec:Vm-clamp}

When switching from one value to another, in analog system, it takes a certain
amount of time call ``ramping time''. In a digital system, this ramping time is
zero. 

METHOD 1: So, in Voltage clamp, if we assume ramping time is zero, we just
switch $V_m$ from $V_\rest$ to $V_\stim$ (ON simulation), and from $V_\stim$
back to $V_\rest$ (OFF simulation).


METHOD 2: Using analog-based concept, the Volage increase from $V_\rest$ to
$V_\stim$ within a time period called ramping time \verb!ramp_time=1! (ms). 


However, this may cause instability in numerical simulation. One protocol being
used is suggested in \citep{santana1996}: Cells were held at -80 mV. Before a
test depolarization, cells were depolarized to -50 mV by a slow
(500-millisecond) voltage ramp and held at -50 mV for 50 milliseconds before
test depolarizations were applied. After the test depolarization, the membrane
potential was returned to -80 mV.



\section{GPU - ECC}
\label{sec:gpu-ecc}

Data mapping to constant memory
\begin{lstlisting}
int_ca = (/ NSFU, N_R, maxnkv1, maxnkv2, maxnkm1, & !5
     maxnkm2, maxnkp1, maxnkp2, maxnkv1+maxnkv2+maxnkm1+maxnkm2+maxnkp1+maxnkp2-5, 0 & ! 10
     /)
!beta_ds : 12 --> 6
    dp_ca = (/ -Ej*0.5d0, Ecc, Eoo, k_jsr0, k_jsr1, & ! 5
     beta_ds, inv_lambda_ds, inv_lambda_jsr, v_ryr, v_refill, & ! 10
     v_efflux, 0.d0, P_dhpr, eta_RyR , eta_LCC, &  ! 15
     F2zCa2_RT, inv_2FVmyo, 0.341d0*Ca_o, Km_jsr, B_jsrT &  ! 20
 /)
\end{lstlisting}

Device memory
\begin{verbatim}
RYRgate_dev, 
U_Ro_dev
I_dhpr_dev
indexPdt_dev

# new
DHPRgate_dev
U_Lo_dev
J_ryr_dev
\end{verbatim}

\section{Gain}
\label{sec:gain}

There are two types of gain: integral gain and peak gain.

The integral gain (\verb!INTEGRAL_GAIN!) is defined as 
\begin{equation}
  \label{eq:50}
  \text{INTEGRAL GAIN} = \frac{\int J_\ryr^T dt}{\int J_\dhpr^T dt}
\end{equation}
An approximate formula is
\begin{equation}
  \label{eq:51}
  \text{INTEGRAL GAIN} = \frac{\sum J_\ryr^T}{\sum J_\dhpr^T}
\end{equation}

The peak gain (\verb!PEAK_GAIN!) is the ratio of the peak of Jryr flux over the
peak of Jdhpr flux.



\section{Integrated RyR Flux}
\label{sec:mass_Ca_via_RyR}

To tell the amount of $\Ca$ ($\mu$M) released via RyR during a certain time, we
use $\int J_\ryr dt$ which can be done numerically as
\begin{lstlisting}
accumulatedRyRCa = 0
Jryr_prev = 0
time_prev = time
do simulation loop
   if (time_prev - time .ge. check_point) then 
      accumulatedRyRCa += (Jryr_prev + Jryr)/2 * (time - time_prev)
      time_prev = time
      Jryr_prev = Jryr
   endif
enddo  
\end{lstlisting}

\section{Data I/O - HDF5}
\label{sec:ecc_data_io}

The data structure is similar to that being discussed for ``leak'' model
(Sect.\ref{sec:leak_hdf5}). 


The structure of the file is given
\begin{verbatim}
  TYPE(HDFFILE) :: file_struct
  
  INTEGER, PARAMETER :: MAX_NUM_HANDLE = 10
  TYPE HDFFILE
     INTEGER(HID_T) :: file_ID
     INTEGER(HID_T), DIMENSION(MAX_NUM_HANDLE) :: arr_dspace ! (data) space
     INTEGER ::  num_dspace
     INTEGER(HID_T), DIMENSION(MAX_NUM_HANDLE) :: arr_dset ! dataset
     INTEGER :: num_dset
     INTEGER(HID_T), DIMENSION(MAX_NUM_HANDLE) :: arr_plist !property list
     INTEGER :: num_plist
  END TYPE HDFFILE
\end{verbatim}

\begin{enumerate}
  \item To create a file
  \begin{verbatim}
CALL h5fcreate_f(TRIM(filename), H5F_ACC_TRUNC_F, file_ID, hdf_error, &
         H5P_DEFAULT_F.H5P_DEFAULT_F)
file_struct%file_ID = file_ID
  \end{verbatim}
  
  
\end{enumerate}

\section{Restart file}
\label{sec:ecc_restart_file}


\section{Detect Cluster State Index}

The function \verb!get_index_XXXcluster_numXXXopen!(ii) return the index from
that we know that there are $ii$ channel opens.

\begin{verbatim}
    !! this data is being used for caffeine OR trigger RYR simulation 
    IF (.NOT. ALLOCATED(index_RYRcluster_numRYRopen)) ALLOCATE(index_RYRcluster_numRYRopen(0:N_R))
    CALL get_index_XXXcluster_numXXXopen(N_R, irow_R, NSFU, U_Ro, &
         index_RYRcluster_numRYRopen)

    !! this data is being used for trigger LCC simulation 
    IF (.NOT. ALLOCATED(index_LCCcluster_numLCCopen)) ALLOCATE(index_LCCcluster_numLCCopen(0:N_L))
    CALL get_index_XXXcluster_numXXXopen(N_L, irow_L, NSFU, U_Lo, &
         index_LCCcluster_numLCCopen)
\end{verbatim}

\section{LCC model (Sun et al. - 2000)}

\begin{verbatim}
! association/dissociation constant Ca-CM
kp_CM = 2.4d1 ! 1/(uM.s)
km_CM = 38d0  ! 1/s
\end{verbatim}
