\chapter{makeKs\_utility.f95}
\label{chap:makeKs_util}

\section{getcompK\_\ldots}
\label{sec:getcompK}

\begin{enumerate}
  \item \verb!getcompK! : premature version, work with 2-state RyR
  \begin{lstlisting}
  getcompK(ak,compK, indexK,m,irows,maxnkl,iUi)
     ak: transition matrix of single channel (size: m x m)
     compK, indexK : compact form (size: compK(irows, 0:maxnkl), 
                                         indexK(irows, maxnkl)
     m = number of state of a single channel
     maxnkl : columns of compK, indexK
     iUi  : vector tells the number of RyRs in each state (size: irows x m)
  \end{lstlisting}
  The number of columns: maxnkl = m * (m-1). The zero-th column has not been
  used.
  
  \item \verb!getcompK2!: a better version of \verb!getcompK! with better
  constrain for maxnkl = number of rows with non-zero elements
  \begin{lstlisting}
  compK(irows, 0:maxnkl)
  indexK(irows, maxnkl)
  \end{lstlisting}
  The negative value is at last column
  \begin{verbatim}
         if (asum .gt. EPS) then
          compK(ii, maxnkl) = -asum
          indexK(ii, maxnkl) = ii
       endif
  \end{verbatim}
  
  The transpose version \verb!getcompK2_T! has
  \begin{lstlisting}
  compK(0:maxnkl, irows)
  indexK(maxnkl, irows)
  \end{lstlisting}
  with negative value is at first row
  \begin{verbatim}
         if (asum .gt. EPS) then
          compK(1, ii) = -asum
          indexK(1, ii) = ii
       endif
  \end{verbatim}  
  The zero-th column/row is still dummy. 
  
  \begin{framed}
  If we use the transpose form, with columns represent the current states
  \begin{verbatim}
  jj = 1...TOTAL_COLUMN
      compK(jj, ct)
  \end{verbatim}
  with ct can be different for each CRU, compared to
  \begin{verbatim}
  jj = 1...TOTAL_COLUMN
      compK(ct,jj)
  \end{verbatim}
  As Fortran use column-based indexing, the regular form is better, i.e.
  compK(irow, maxnkp)
  \end{framed}
  
  \item \verb!getcompK_3! : a better version that works with full EC-coupling
  model. From now on, the negative value is kept at the first row/column,
  depending whether transpose version is used or not, i.e.
  \begin{verbatim}
          compK(ii, 1) = -asum
          idxK(ii, 1) = ii
 or
           compK(1, ii) = -asum
          idxK(1, ii) = ii 
  \end{verbatim} 
	Also, the constraint for the columns (i.e. maximum number of non-zero state
	transition) is the number of non-zero non-diagonal elements. 
Now, the index matrix has the zero-th column/row.
\begin{verbatim}
    ALLOCATE(idxK(irows, 0:maxnkl))
\end{verbatim}
which is set to the true number of neighbors at the corresponding state index
\begin{verbatim}
       IF (icount > 1) THEN
          idxK(ii, 0) = icount
       ELSE
          idxK(ii, 0) = +0
       ENDIF
\end{verbatim}

\item \verb!getcompK_4! : a similar version of \verb!getcompK_3!, yet give
better column constraint when the number of channel is less than the number of
state for a single channel. 
\end{enumerate}


\section{makeKs}
\label{sec:makeKs}

\begin{enumerate}
  \item \verb!makeKs()! : create the transition matrices, in full format, for
  the regular release site unit (CRU)
  \begin{lstlisting}
    SUBROUTINE makeKs(aKv1,aKv2,aKm1,aKp1,aKm2,aKp2,U_Lo,U_Ro,N_L,N_R,irow_L,irow_R)
  \end{lstlisting}
	It uses \verb!getbigK()! subroutine. 
  
  \item \verb!makecompKs()!: the first version of compact form matrices. This
  work with Leak study as it generates the compact form for RyR cluster only.
\begin{lstlisting}
  SUBROUTINE makecompKs(compKm, compKp, indexKm, indexKp, U_Ro, N_R &
       ,irow_R,maxnklm, maxnklp)
\end{lstlisting}
which call \verb!setupRyR_2state!
\begin{verbatim}
    CALL setupRyR_2state(N_R, &
         compKm, compKp, &
         indexKm, indexKp, &
         U_Ro, irow_R, maxnklm, maxnklp, "leak")
\end{verbatim}
Correspondingly, we also have the transpose version \verb!makecompKs_T!.

\item \verb!makecompKs_2()! : the first version of compact form matrices that is
designed for a heterogeneous cluster (LCC + RyR). 
\begin{lstlisting}
  SUBROUTINE makecompKs_2(compKv1, compKv2, compKm1, compKm2, compKp1, &
       compKp2, idxKv1, idxKv2, idxKm1, idxKm2, idxKp1, &
       idxKp2, U_Lo, U_Ro, N_L, N_R, irow_L, irow_R, maxnkv1, maxnkv2, maxnkm1, &
       maxnkm2, maxnkp1, maxnkp2)
\end{lstlisting}
The transition matrix for the heterogeneous clusters is formed by using either
\begin{verbatim}
    CALL getclustercompKleft_T(compK_Lv1, idxK_Lv1, &
         irow_L, irow_R, maxnkLv1, compKv1, idxKv1, maxnkv1)
\end{verbatim}
or
\begin{verbatim}
    CALL getclustercompKright_T(compK_Rp, idxK_Rp, &
         irow_L, irow_R, maxnkRp, compKp2, idxKp2, maxnkp2)
\end{verbatim}
LIMITATION: This function is designed to work with 6-state LCC + 2-state RyR. 

\item \verb!makecompKs_3()!: This function is designed to work with any model of
LCC + RyR.
\begin{lstlisting}
  SUBROUTINE makecompKs_3(compKv1, compKv2, compKm1, compKm2, compKp1, &
       compKp2, idxKv1, idxKv2, idxKm1, idxKm2, idxKp1, &
       idxKp2, U_Lo, U_Ro, N_L, N_R, irow_L, irow_R, maxnkv1, maxnkv2, maxnkm1, &
       maxnkm2, maxnkp1, maxnkp2, setupLCC, setupRyR)
\end{lstlisting}

\item \verb!makeKs_nj_6_2!: full matrices for single channel(6-state LCC +
2-state RyR), as non-junctional channels are modeled deterministic.
\begin{lstlisting}
  SUBROUTINE makeKs_nj_6_2(aKv1, aKv2, aKm1, aKp1, aKm2, aKp2, U_Lo, &
       U_Ro, pL_nj, pR_nj, irow_L, irow_R) 
\end{lstlisting}

\verb!makeKs_nj_7_2!: work with 7-state LCC + 2-state RyR

\item \verb!makeKs_orphan_2! = \verb!makeKs_RyRsmall_2! : generate transition
matrices in full form of a cluster of RyR per se. This is used to model
orphaning RyR or a small size RyR. 


\item \verb!makecompKs_RyRsmall_2!: 
\begin{verbatim}
CALL makecompKs_RyRsmall(N_R_small, compK_Rm_small, compK_Rp_small, &
           idx_Rm_small, idx_Rp_small, &
           U_Ro_small, irow_R_small, maxnkRm_small, maxnkRp_small, setupRyR_2state)
\end{verbatim}
The transpose form is \verb!makecompKs_RyRsmall_2_T!.
\end{enumerate}