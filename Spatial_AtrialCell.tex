\chapter{Spatial Model Atrial Cell}
\label{chap:spatial_atrial_cell}


There are major change in coding the model for atrial cells. This is mainly
caused by the fact that in atrial cell, the microdomain of subspace is very
rarely (10-20\%). So most of the time, the ryanodine receptor (RyR2) see the calcium in the myoplasm. 
To study the dynamics of calcium sparks, in the model, we removed this effect by
making some changes in the temporalspatial code.

\begin{enumerate}
  \item  \verb!Ca_ds_dev! is removed from the code, instead we replace it with
  \verb!Ca_myo_in(offset)! with \verb!offset=SFU_in_lingrid_dev(ii)! with $ii$
  is the index of the 'pseudo'-SFU in which we still use to detect the jSR location. 
 
 \item As a matter of fact, the calcium from $I_\dhpr$ and $J_\ryr$ flow directly into
the grid point containing myoplasm. So, we don't need to have
\verb!J_efflux! nor \verb!J_efflux_part! in the code.

\begin{verbatim}
J_efflux_dev(ii) = J_dhpr + J_ryr_dev(ii)
J_efflux_part_dev(offset) = J_efflux_dev(ii)
\end{verbatim}

 \item Using of \verb!_CRU_LAYOUT_STRATEGY_! macro is removed. Also, the flux
 rate need to be adjusted
\begin{verbatim}
v_ryrT = ???
P_dhprT = ???
\end{verbatim}
NOTE: This is permeabilized cell, so we run \verb!_MODE_CLOSED_CELL! simulation,
and we don't use \verb!P_dhprT! value.

  \item The detection of the peak of the spark is now using
  \verb!Ca_peak_prev_dev! rather than \verb!Ca_ds_prev_dev!. 
  
  \item Some functions has been modified to adapt to atrial cell
  \begin{verbatim}
  analyze_Sparks_2(... , Ca_myo_in)
  analyze_Sparks_3(... , Ca_myo_in)
  \end{verbatim}
  
  \item New macro to determine blocking SERCA or not
  \begin{verbatim}
  #define _USE_SERCA_ _YES
  \end{verbatim}
  
  \item New macro to incorporate the effect of EGTA buffer
  \begin{verbatim}
    #define _USE_EGTA_ _YES
  \end{verbatim}
\end{enumerate}

The code works with grid-point of size 100nm and 50nm. 
\begin{enumerate}
  \item 49RYR: it senses calcium from 300x200x100 (which uses macro
  \verb!_3x2x1! or \verb!_6x4x2!, depending on the grid-size being used)
  \item 25RYR: it senses calcium from 200x200x100 (which uses macro
  \verb!_2x2x1! or \verb!_4x4x2!)
\end{enumerate}

\textcolor{red}{All simulations were done under the condition of permeabilized
cells}, i.e. no change in cytoplasmic calcium. The exogenous buffer EGTA was
incorporated to avoid calcium waves.

The major change in AF compared to Ctrl is the buffer changes, and an increase
in the number of sattelite clusters. In Ctrl, it has about 2 satellite clusters,
at a distance within 150nm that form the super-cluster or a functioning group
with the center cluster. In AF, it has about 4 satellite clusters. Also,
macrosparks are observed more frequently in AF (from very rarely in Ctrl to
about 10\% in AF), a simulation using 50nm grid-size was done. 


\section{Macro-sparks}

We do two things to test macro-sparks

\begin{enumerate}
  \item One big cluster (384RyRs), the activation is 
\begin{verbatim}
//macro.h
#define _ARRANGE_CRU_ _CRU_UNIFORM_WITH_ROGUE_RYR
#define _CRU_LAYOUT_STRATEGY_ _9x9x1
#define _EMULATE_CONDITION_ _CUSTOMIZED_CRUS
\end{verbatim}  
and no rogue RyR setting.

  \item Use a customize CRU arrangement setting  
\begin{verbatim}
//macro.h
#define _ARRANGE_CRU_ _CRU_MACROSPARK_WITH_ROGUE_RYR

//tscell_gpu_utility.f95
  #elif _ARRANGE_CRU_ == _CRU_MACROSPARK_WITH_ROGUE_RYR
    ierr = tscell_layout_CaRUs2grid_linear(NSFU, SFU_in_lingrid, lingrid_with_SFU, &
         put_SFU2grid3D_macrospark_rogue_RYR)

\end{verbatim}
\end{enumerate}


Also, the activation of the cluster is depending on the calcium highest among
the grid-points covered by the RyR cluster, rather than the averaged calcium
level at these grid-points.
\begin{enumerate}
  \item 9x9x1
\begin{verbatim}
!!9x9x1
       s_Ca_myo(tx) = 0.0d0
       do mi = -4, 4
             dtemp = maxval(Ca_myo(offset+mi*int_ca(17)-4 : offset+mi*int_ca(17)+4))
             s_Ca_myo(tx) = max(s_Ca_myo(tx), dtemp)
       end do
\end{verbatim}

  \item 3x3x1
\begin{verbatim}
!!3x3x1
       s_Ca_myo(tx) = 0.0d0
       do mi = -1, 1
             dtemp = maxval(Ca_myo(offset+mi*int_ca(17)-1 : offset+mi*int_ca(17)+1))
             s_Ca_myo(tx) = max(s_Ca_myo(tx), dtemp)
       end do
\end{verbatim}  
\end{enumerate}


