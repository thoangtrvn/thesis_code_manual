\chapter{Plotting code for Data analysis}

The code is written in IDL. Timeline
\begin{enumerate}
  \item Before Oct, 31: the main plot function can be 
  \begin{verbatim}
  main_plot.pro
  main_plotspark.pro
  main_plotspark_sep19.pro
  main_plotspark_oct6.pro
  \end{verbatim}
  \item From Nov, 1st, 2012: the main plot function is 
  \begin{verbatim}
  main_plot_nov01.pro
  \end{verbatim}
  The new feature is plotting ionic currents-time, and $V_m$-time plots.
\end{enumerate}

At first, we need to add search path. Here we use Coyote IDL Library and
Astronomy IDL library
\begin{verbatim}
SEARCH_PATH = Expand_Path('+~/Projects/coyote_library_1.10.0') + $
  ':<IDL_DEFAULT>'+':./:'+ Expand_Path('+~/Projects/astron/pro/') + $ 
  ':' + Expand_Path('+~/Projects/idl_routines/') 
pref_set, 'IDL_PATH', SEARCH_PATH, /COMMIT
\end{verbatim}


Define global variables in different COMMON group-names
\begin{verbatim}
COMMON block_paramfiles, NSFU, N_R, N_L, mL, mR, $
     V_rest, V_stim, I_stim, t_start, t_end, $
     t_V_stim_start, t_V_stim_end, t_I_stim_start, t_I_stim_end, $
     Hz, write_interval, $ ;Na_i, K_i,
     Ca_myo, Ca_nsr, $
     zF, zR, zT, Na_o, Ca_o, K_o, V_myo_T, V_nsr_T, $
     V_jsr_T, V_ds_T, beta_ds, beta_nsr, B_myoT, $
     B_jsrT, Km_myo, Km_jsr, B_htrpnT, kon_htrpn, $
     koff_htrpn, v_refill_T, v_efflux_T, v_leak, $
     v_ryr_T, P_dhpr_T, prcent_ryr_nj, prcent_dhpr_nj, $
     Ap, Kp_myo, Kp_nsr, g_bCa, g_Na, g_K1, $
     g_Kss, g_Ktof, g_Ktos, g_bNa, g_bK, $
     Am, Csc, I_pmca_bar, I_ncx_bar, I_NaK_bar, $
     eta_RyR, eta_LCC, Ej, Ecc, Eoo, k_jsr0, $
     k_jsr1, V_theta1, V_sigma1, V_theta2, V_sigma2, $
     cell_x_len, cell_y_len, cell_z_len, dX_, dY_, dZ_, $
     Dmyo_x, Dmyo_y, Dmyo_z, Dnsr_x, Dnsr_y, Dnsr_z, $
     lspacing, tspacing, dist2membrane, dist2rogueRyR, $
     dist2membraneX, dist2membraneYZ, $  
     num_rogueRyR, Ddye_x, Ddye_y, Ddye_z, F_T, kon_F, koff_F, $
     write_firstbeat, duration_write_lastbeats, write_restartfile, kR12, $
     N_R_rogue, i_1ryr, caf_factor
COMMON block_deriveddata, zCa, eta_PMCA, Km_pmca, eta_NCX, $
     Ksat_ncx, Km_ncxNa, Km_ncxCa, cpl_serca, RT_F, F_RT, RT_FzCa, $
     P_dhpr, V_ds, V_jsr, lambda_ds, lambda_jsr, lambda_nsr, $
     inv_lambda_ds, inv_lambda_jsr, inv_lambda_nsr, Acap, $
     Am_2FVmyo, Am_FVmyo, v_refill, v_ryr, v_ryr_nj_T, P_dhpr_nj, $
     F2zCa2_RT, FzCa_RT, invKp_myo, invKp_nsr, inv_2FVmyo, $
     NUM_STATES_NJ, NUM_STATES_LCCs, NUM_STATES_RyRs, $
     ULo, ULc, ULr, ULi, ULa, ULo_nj, URc, URo, NSFU_normal, Kd_F
  ; CRU_ID = index of the CRU to plot in spatial_singlespark
  ; base_filename = base filename of data to read ('cmyo-', 'cnsr-', 'CaF-')
COMMON block_globalvars, MAX_NUM_DATASET_PER_GROUP, CRU_ID, $
     STENCIL_RADIUS, x_stride, y_stride, z_stride, x_safe, y_safe, $
     z_safe, glob_flag_write2png, base_filename
COMMON block_datafolder, target_folder, root_dir, username, sub_folder
COMMON block_greekchar, char_mu, char_delta, char_bigDelta
COMMON block_tekcolor, black_col, red_col, blue_col, green_col, cyan_col, magenta_col
  ;; variables that keeps information of plot positions and settings
COMMON block_plotsystem, plots_sysp, plots_sysx, plots_sysy, plots_loc, time_range, text_pos_x, text_pos_y, main_string
\end{verbatim}
NOTE:
\begin{enumerate}
  \item \verb!i_1ryr, caf_factor!: added Sep, 21, 2012
  
\end{enumerate}

Initial values setting
\begin{verbatim}
  black_col = 0
  red_col = 2
  blue_col = 4
  green_col = 3
  cyan_col = 5
  magenta_col = 6
  
  MAX_NUM_DATASET_PER_GROUP = 100
  STENCIL_RADIUS = 1
  glob_flag_write2png = 0
  glob_file_index = 0 ;; starting index of data files
  username = 'minhtuan'
  main_string = []
  param_file = 'parameters.conf'
  root_dir = '/leak1/'
  target_folder =  'tscell_rogue_2012jun21i'
  sub_folder = ''  ; or 'firstbeat' (a subfolder inside the target_folder)
  CRU_ID = 3
  ;;; the typical pixel size is 0.14um x 2ms
  ;;; i.e. 
  pixel_resolution = 0.14d0      ; um (0.14um)
  scanningtime_interval = 0.5d0  ;; ms (default: 2.0)
\end{verbatim}

The main function to call is \verb!main_plotspark!, but we should make new copy,
adding the date and the description for what type of analysis being called in
the file.

\section{Make line-scan image}

We have 4 versions, all using PSF in 3D, and the result is CaF
\begin{verbatim}
make_linescan_image_psf3d
make_linescan_image_psf3d_v2
make_linescan_image_psf3d_v3
make_linescan_image_psf3d_v4
\end{verbatim}

We can also make line-scan image for free calcium using (not sure about PSF
2D/3D???)
\begin{verbatim}
make_linescan_image_cmyo
\end{verbatim}

\section{Analyze rogue - normal cluster}

\verb!SpatialCell_GPU_branch2_smallcluster_0.20um/plot_code/!

In the case of triggering normal CRU and analyze rogue RyRs, we use
\begin{verbatim}
analyze_rogueryr
\end{verbatim}

In the case of triggering rogue RyR and analyze normal CRU, we use
\begin{verbatim}
analyze_rogueryr_triggerrogue
\end{verbatim}


\section{Read-in Data}

\subsection{read\_hdf\_regular}
\label{sec:read_hdf_regular}

\verb!read_hdf_regular! 
\begin{verbatim}
pro read_hdf_regular, hdf_filename=hdf_filename, type=type                                                                                                                                                 
  ;; type can be 'general', 'sfu-state', 'cds', 'cjsr'    
\end{verbatim}

This is later renamed to \verb!parse_hdf_regular!
(Sect.\ref{sec:parse_hdf_regular}).

\subsection{parse\_hdf\_regular}
\label{sec:parse_hdf_regular}

This is the newer version of Sect.\ref{sec:read_hdf_regular}.

