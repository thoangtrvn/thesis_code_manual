\chapter{General setting and conventions}
\label{chap:fortran-code-tuan}

\def\Na{{\text{Na}^+}}
\def\Cl{{\text{Cl}^-}}
\def\inj{{\text{inj}}}

\section{Introduction}

This document was written as a supplement to the Fortran code developed by Tuan
M. Hoang-Trong as part of his PhD thesis on modelling cardiac ventricular
myocyte. The model started with LEAK code (Chap.\ref{sec:ca2+-leak-}) which
simulates the gating of RyR ion channels in thousands of release site units.
The code was then extended to model the stochastic EC-coupling of the rat
ventricular myocytes (Chap.\ref{sec:ec-coupling}). In the two code above, both
CPU and GPU versions are available.

A 3D model of the rat ventricular myocytes was then developed using
GPU-technology, i.e. no CPU version for this code. There are two variants of the
code was developed. The first code was designed with no rogue-RYR (satellite
RYR) (Chap.\ref{chap:spatial-cell}) and then with rogue-RYR
(Chap.\ref{chap:small_RyR_cluster}). One code for atrial cell was also developed
(Chap.\ref{chap:spatial_atrial_cell}).

\section{Unit conversion}
\label{sec:unit-conversion}

NOTE: concentration: Molar (M) = mole/litre, uM = $\mu$mol/litre
\begin{eqnarray}
  \label{eq:4}
  1 \mol = 6.02 \times 10^{23} \text{atoms} \\
  \text{M = mol/L} = 10^6 \mu \text{mol/L} = 10^6 \mu\text{M}
\end{eqnarray}

NOTE: 1 Jule = work to move an electric charge (1Coulomb) through an
electric potential difference of 1Volt, or 1 (J) = 1 (C.V)
\begin{equation}
  \label{eq:7}
  1\text{ (mJ)} = 1 \text{ (C.mV)}
\end{equation}

Here, we assume the current is $u$A/cm$^2$.
We can use  $\mu$A/cm$^2$ or $\mu$A/$\mu$F interchangebly as the
specific membrane capacity is $\Csc=1.00\mu$F/cm$^2$.

\begin{framed}
  NOTE: 1 Ampere = 1 (A) = 1 (C/sec) = (F.V/sec)

  NOTE: 1 Siemen = 1 (S) = 1 (A/V)
\end{framed}



All volume need to be converted into (pL).
\begin{verbatim}
1 (pL) = 1d-12 (L) = 1d-6 (uL) 
    = 1d-12 (dm^3) = 1d-15 (m^3) 
    = 1d3 (um^3) = 1d-9 (cm^3)
1 (L)  = 1 (dm^3) = 1d3 (cm^3) 
\end{verbatim}

NOTE: 1 Coulomb = 1 (C) = 1 (Ampere.sec) = 1 (A.sec) = 1 (Fara.Volt) =
1 (F.V)

\begin{verbatim}
1 (cm) = 1e-2 (m) = 1e-2x1e6 (um) = 1e4 (um) 

1 (cm^2) = 1e-4 (m^2) = 1e-4x1e12 um^2 = 1e8 (um^2)

1 (cm^3) = 1e-6 (m^3) = 1e-6x1e18 um^3 = 1e12 (um^3)

1 (m)  = 1d9 (nm) = 1d6 (um)
1 (dm) = 1d8 (nm) = 1d5 (um)
1 (L)  = 1 (dm^3) = 1d24 (nm^3) = 1d15 (um^3)
1 (pL) = 1d-12 (L) = 1d-12 (nm^3) = 1d3 (um^3)
\end{verbatim}

\subsection{Standard units }
\label{sec:standard-units-}

Using the convention
\begin{equation}
  \label{eq:25}
  \Csc\frac{dV_m}{dt} = -( I_\ion + I_\stim) 
\end{equation}
with the unit of $\Csc$ is $\mu$F/cm$^2$, and $V_m$ is mV; while
$I_\ion$ is in $\mu$A/cm$^2$. Then quantitatively, we need to convert
$\Csc$ to mF/cm$^2$, by using

\begin{equation}
  \label{eq:49}
  10^{-3}\Csc\frac{dV_m}{dt} = -( I_\ion + I_\stim)   
\end{equation}

with the ionic current density ($\mu$A/cm$^2$)
\begin{equation}
  \label{eq:26}
  I_\ion = I_\Na + I_\Kr + I_\Ks + I_\CaL + ...
\end{equation}
%All the ionic currents are defined based on $V_\myo^T$. 

\begin{framed}
Membrane currents and membrane conductance are often normalized to the
cell membrane capacitance, i.e. pA/pF. This approach takes advantage
of the fact that ``specific capacitance'' of cell membrane is
$\Csc\sim 1\mu$F/cm$^2$ which is fairly constant among different
muscle cell types and species.  So, the cell capacitance is often used
to estimate the surface membrane area of the cells, and an indirect
index to cell volume. 
In other words, the density can be defined based on a unit of membrane area or a
unit of membrane capacity. The eq.\ref{eq:26} use the first approach. The
following formula use the latter apporach, i.e. some literatures use 
  \begin{equation}
    \label{eq:27}
    \frac{dV_m}{dt} = -( I_\ion + I_\stim) 
  \end{equation}
  where the specific membrane capacitance $\Csc$ is now incorporated
  into the ionic current, so the ionic current density is in unit
  ($\mu$A/$\mu$F) (or pA/pF), rather than current per unit membrane area,
  e.g. ($\mu$A/cm$^2$). Quantitatively, there is no change in the value of
  individual ionic current when changing from [ampere/$\mu$F] to
  [ampere/cm$^2$], as $\Csc=1\mu$F/cm$^2$. That's why this is the source of
  mistakes when reading papers using different conventions.
\end{framed}


\begin{enumerate}
\item Membrane potential: (mV)
\item Membrane (ionic) current: ($\mu$A/cm$^2$)  (or some models use (pA/pF) or (uA/uF))
\item Membrane conductance: (mS/cm$^2$)          (or some models use (mS/uF))
\item Ionic fluxes: (uM/sec) or (mM/ms)
\item Concentration: (uM)
\item Time constant: (s)
\item Rate constant: (s)
\item Stimulus current: (nA), e.g. -1 (nA) is the max that most equipments
  can do, $I_\stim=\frac{-1\times 10^{-3}}{\Am}$ (uA/cm$^2$), with $\Am$ is the
  plasma membrane area. 
\end{enumerate}

\textcolor{red}{However, there are some current that're being
  calculated at whole-cell level, e.g. $I_{\dhpr,\nj}$ and
  $I_\dhpr^\ii$, where the unit is ($\mu$A)}.





\section{Current (pA) and number of ions (moles)}

1(pA) is $10^{-12}$ (A) = $10^{-12}$ (Coulomb/sec).

\begin{enumerate}
  \item 1 mole of electrons (-e) carries \verb!F = 96,485! Coulombs of electric
  charge
  
NOTE: F = 96,485 (Coulomb/mole) is Faraday constant.
  
  \item 1 mole of univalent ions (+e or -e), such as $\Na$, $\Cl$, carries
  \verb!F! Coulombs of electric charge
  
  \item 1 mole of divalent ions (e.g. $\Ca$), carries $(z_\text{ion} \times F)$
  Coulombs of electric charge.
  
NOTE: $z_\text{ion}=2$ is the valence of the divalent ion.
\end{enumerate}

So, 1 mole of any ion carries $(z_\text{ion}\times F)$ Coulomb of electric
charges.
In other words, 1 Coulomb of electric charge is carried by
$\frac{1}{z_\text{ion} \times F}$ mole of such ion.

So, $I$ (pA) of $\Ca$ ions means the transfer rate is
$\frac{10^{-12}\times I}{z_\ca F}$ (mole/sec) or $\frac{10^{-6}\times I}{z_\ca F}$ ($\mu$mole/sec)
or $\frac{10^{-9}\times I}{z_\ca F}$ ($\mu$mole/msec).
If such ion flux transfer to a region of volume $V$, we can convert to
concentration flux, 
\begin{equation}
\frac{10^{-6}\times I}{z_\ca \times F \times V} \qquad \text{($\mu$M/sec).}
\end{equation}


\section{Convert current flux (pA/cm$^2$) to concentration flux (mM)}
\label{sec:convert-current-density-to-concentration}

Recall that one molar is one mole per liter or one mole per 1000 cubic
centimeters.

Faraday's constant, 96,485 Coulomb/mole, means the number of Coulomb per mole of
electron.

\subsection{Current density $I_x$ ($\mu$A/cm$^2$) to concentration flux $J_x$
(mM/msec)}

Input: current $I_X$ ($\muA$/cm$^2$).

% Suppose the valance of the ion is $z$. Then $\frac{I_X}{(zF)}$ gives us
% the transmembrane flux in units of $\muM$/cm/s.
% To convert this into a concentration flux, we suppose that the ions collect in a
% thin layer of depth d (in microns, i.e. $\mum$) near the surface of the cell.
% Thus, the change in concentration is $I_X/(zdF)$ .

We study the change of ions $X$ collected in a volume  before the membrane, i.e.
a thin layer of depth $d$, is
\begin{equation}
J = \frac{I_x}{z_x . F . d} = \frac{10 I_x }{z_x.F.d} \qquad
\text{[mole/(sec.L)]}
\end{equation}
with F = 96,485 Coulomb/mole, d (in $\mum$).
Another equivalent unit for $J$ is [mM/msec].


Output: mM/msec = millimoles per liter per millisecond.


SUMMARY: The total in(out)flux of ions across the thin volume layer
\begin{equation}
f_X = 10 I_X / (z.F.d)
\end{equation}
with F = 96,485 (C/mole), $d$ (depth: $\mum$), $I_X$ (mA/cm$^2$).

\subsection{Current density $I^\cdot_x$(pA/$\mum^2$) to total current (pA)}

Suppose the cross-sectional area is $A$ ($\mum^2$), then the total current I
(pA) is
\begin{equation}
I_\inj = I^\cdot_X \times A   \qquad \text{(pA)}
\end{equation}

%\subsection{Total current (pA) to concentration }
%\subsection{$I_x$ ($\mu$A/cm$^2$) to $J_x$ (mM/msec)}


\section{Cell Volume, Cross-sectional area, and Surface membrane area}
\label{sec:cell_volume}

The study to direct the cross-sectional cell area to the cell volume
is given by Satoh et al. (1996). Say for rat
\begin{verbatim}
Cell Volume (pL) = 7.59x1d-3 (pL/um^2) * cross-sectional-area
\end{verbatim}
Between cell volume and sarcolemma surface area, we use the
formula~\citep{page1978}
\begin{verbatim}
surface-area = Vol (pL) * 0.5 (1/um)
\end{verbatim}
or
\begin{verbatim}
surface-area (cm^2) = (((Vol (pL) * 1e3) [um^3] * factor (1/um)) * 1e-8) (cm^2)
\end{verbatim}
with \verb!factor! is about 0.5. In our case, we use 0.426 which give surface
area as 1.534d-4 (cm$^2$) from Vol=36 (pL).


% However, we don't know the direct
% relation between cell volume and sarcolemma surface area. Maybe a
% simple rescale with the same ratio is good enough?

\section{Buffers}
\label{sec:buffers}

Buffers are proteins or species that $\Ca$ bind to (1) reduce the amount of free
calcium, (2) perform calcium signalling. In subspace, calcium can bind
to (1) calmodulin (CaM), (2) sarcolemma, and (3) SR.

In myoplasm, calcium can bind to (1) calmodulin (CaM), and (2) troponin (high
and low). Typically, Troponin is modeled as explicit buffer, and CaM is modeled
as fast buffer.

In JSR, a large amount of calcium is bound to calsequestrin (CSQ2) and is
modeled as fast buffer. 

In NSR, CaMKII is the buffer ...

% \section{Fast buffering}
% \label{sec:fast-buffering}

Using fast buffering, the buffering fraction
\begin{verbatim}
## CaM-liked buffer
beta_myo                                                [unitless]
         = 1.0d0 /(1.0d0 + (B_myoT * Km_myo)/((Km_myo + Ca_myo)**2))
\end{verbatim}
\begin{equation}
  \label{eq:20}
  \beta_\myo = \left( 1+ \frac{[\B]_\myo^T.K_{m,\myo}}{(K_{m,\myo}+[\Ca]_\myo)^2}\right)^{-1}
\end{equation}
\begin{verbatim}
## CSQ2-like buffer
beta_jsr 
       = 1d0/( 1d0 + (B_jsrT * Km_jsr)/((Km_jsr + Ca_jsr(ii))**2))               
\end{verbatim}
\begin{equation}
  \label{eq:21}
  \beta_\jsr = \left(1+ \frac{[\B]_\jsr^T.K_{m,\jsr}}{(K_{m,\jsr}+[\Ca]_\jsr)^2}\right)^{-1}
\end{equation}
Buffering fraction in the NSR, and subspace is assumed to be fixed (see
\verb!beta_ds!, \verb!beta_nsr!). However, we can also use dynamics form with
\begin{equation}
\beta_\ds = \left( 1+
\frac{[\B]_\myo^T.K_{m,\myo}}{(K_{m,\myo}+[\Ca]_\ds)^2}\right)^{-1}
\end{equation}


\section{Balance equations}
\label{sec:balance-equations}

\subsection{Non-spatial}
\label{sec:balance-eq_non-spatial}

The concentration of calcium in each component is defined based on the volume of
that component. To account for the balance of calcium in a closed system, we
have the net change in number of moles should be equal to zero at steady-state
condition (Sect.\ref{sec:steady_state})
\begin{equation}
  \label{eq:5}
  \begin{split}
    V_\myo^T \frac{d[\Ca]_\myo}{dt} + \sum^\NSFU 
  \left(V_\ds\frac{d[\Ca]_\ds}{dt}\right) + \sum^\NSFU
  \left(V_\jsr\frac{d[\Ca]_\jsr}{dt}\right) + \\
  V_\nsr^T
  \frac{d[\Ca]_\nsr}{dt} + V_\myo^T \frac{d[\CaBhtrpn]}{dt} + V_\myoT
  \frac{d[\CaF]}{dt} = 0
  \end{split}
\end{equation}
% or
% \begin{equation}
%   \label{eq:6}
%   V_\myo^T \frac{d[\Ca]_\myo}{dt} + \sum 
%   \left(V_\ds\frac{d[\Ca]_\ds}{dt}\right) + \sum
%   \left(V_\jsr\frac{d[\Ca]_\jsr}{dt}\right) + V_\nsr^T
%   \frac{d[\Ca]_\nsr}{dt} + \frac{d[\Cahtrpn]}{dt} = 0  
% \end{equation}
or
\begin{equation}
  \label{eq:3}
  \begin{split}
     \frac{d[\Ca]_\myo}{dt} + \sum^\NSFU 
  \left(\frac{V_\ds}{V_\myo^T}\frac{d[\Ca]_\ds}{dt}\right) + \sum^\NSFU
  \left(\frac{V_\jsr}{V_\myo^T}\frac{d[\Ca]_\jsr}{dt}\right) + \\
  \frac{V_\nsr^T}{V_\myo^T} \frac{d[\Ca]_\nsr}{dt} + \frac{d[\CaBhtrpn]}{dt} +
  \frac{d[\CaF]}{dt} = 0
    \end{split}    
\end{equation}
By defining volume ratio
\begin{equation}
\lambda_\ds = \frac{V_\ds}{V_\myo^T};\;\;
\lambda_\jsr = \frac{V_\jsr}{V_\myo^T};\;\;
\lambda_\nsr = \frac{V_\nsr^T}{V_\myo^T}
\end{equation}

Given that all fluxes (Sect.\ref{sec:fluxes}) are defined based on \verb!VmyoT!,
then
\begin{verbatim}
dCamyoT   = sum(J_efflux) + J_leak + J_sarcomem + J_ryr_njT -  
            J_serca - J_trpn - J_CaF
J_trpn = dCaBhtrpn
J_CaF  = dCaF     
\end{verbatim}
So, when we calculate the concentration change in other compartments, we need to
map the flux to the correct volume by using effective volume ratio, i.e. either
multiplying with  $V_\myo^T/V_\text{reference volume}$ or dividing
$\lambda_\text{reference volume} = V_\text{reference volume}/V_\myo^T$.
To achieve that balance, the fluxes need to be mapped to the corresponding
volume in the individual compartment, i.e. 
\begin{verbatim}
dCansrT = VmyoT/VnsrT * (J_serca - sum(J_refill) - J_leak - 
            J_ryr_njT)
dCads = VmyoT/Vds * (J_ryr - J_efflux)
dCajsr = VmyoT/Vjsr * (J_refill - J_ryr)
\end{verbatim}

To consider fast buffering effect, we multiply the change with a buffering
factor
\begin{verbatim}
dCamyoT = beta_myo * (sum(J_efflux) + J_leak + J_sarcomem  + 
                      J_ryr_njT - J_serca - J_trpn - J_CaF) 

dCansrT = beta_nsr/lambda_nsr * (J_serca - sum(J_refill) -
                                    J_leak - J_ryr_njT)) 
dCads = beta_ds/lambda_ds * (J_ryr - J_efflux))
dCajsr = beta_jsr/lambda_jsr * (J_refill - J_ryr))
\end{verbatim}
with the buffering factors are given in Sect.\ref{sec:buffers}.

Other Calcium-bound complexes
\begin{verbatim}
dCaBhtrpn = kon_htrpn*Ca_myo*(B_htrpnT - CaB_htrpn) - koff_htrpn*CaB_htrpn  
dCaF = kon_F * (Ca_myo_in(current) * F_in(current)) - koff_F * CaF_in(current) 
\end{verbatim}
% The concentration of any ion (e.g. calcium) in different components is
% defined based on the corresponding volume of that component. So, to
% avoid any conversion, we map them to a single volume, which is the
% myoplasmic volume $V_\myo^T$. If we define the volume fractions as
% \begin{equation}
%   \label{eq:23}
%   \lambda_\nsr = \frac{\hat{V}_\ds}{\hat{V}_\myo^T} = \frac{V_\ds/\beta_\ds}{V_\myo^T/\beta_\myo}
% \end{equation}
% \begin{verbatim}
%     lambda_ds = V_ds/V_myo_T
%     lambda_jsr = V_jsr/V_myo_T
%     lambda_nsr = V_nsr_T/V_myo_T
% \end{verbatim}

\subsection{Spatial}
\label{sec:balance-eq_spatial}

In the compartmental model, the concentrations of the species are defined based
on the volume of the corresponding compartment containing the species. In
spatial model, except calcium in the subspace and jSR, there's a change in the
way we define concentration of calcium in the cytosol and network SR. The volume
reference is for a single grid point, not for the whole compartment. Then,
suppose $V_\myo, V_\nsr$ are the volumes of myoplasm and network SR for a single
grid point, respectively.

Again, we apply the balancing equation
\begin{equation}
\begin{split}
 \sum V_\myo \frac{d[\Ca]_\myo}{dt} + \sum^\NSFU 
  \left(V_\ds\frac{d[\Ca]_\ds}{dt}\right) + \sum^\NSFU
  \left(V_\jsr\frac{d[\Ca]_\jsr}{dt}\right) + \\ 
  \sum V_\nsr
  \frac{d[\Ca]_\nsr}{dt} + \sum V_\myo \frac{d[\CaBhtrpn]}{dt} + \sum V_\myo
  \frac{d[\CaF]}{dt} = 0
\end{split}
\end{equation}

\subsubsection{Referenced volume for the fluxes is VmyoT}

We keep using VmyoT as the reference volume for the fluxes. 
\begin{equation}
\begin{split}
 \sum \frac{V_\myo}{V_\myo^T} \frac{d[\Ca]_\myo}{dt} +  
   \sum\left(\frac{V_\ds}{V_\myo^T}\frac{d[\Ca]_\ds}{dt}\right) + 
  \sum\left(\frac{V_\jsr}{V_\myo^T}\frac{d[\Ca]_\jsr}{dt}\right) + \\
  \sum \left(\frac{V_\nsr}{V_\myo^T}\frac{d[\Ca]_\nsr}{dt} \right)+ \sum
  \frac{V_\myo}{V_\myo^T} \frac{d[\CaBhtrpn]}{dt} + \sum \frac{V_\myo}{V_\myo^T}
  \frac{d[\CaF]}{dt} = 0
\end{split}
\end{equation}
which can be rewritten as
\begin{equation}
\begin{split}
 \sum \frac{1}{\NGRID} \frac{d[\Ca]_\myo}{dt} +  
  \sum\left(\frac{V_\ds}{V_\myo^T}\frac{d[\Ca]_\ds}{dt}\right) + 
  \sum\left(\frac{V_\jsr}{V_\myo^T}\frac{d[\Ca]_\jsr}{dt}\right) + \\
  \sum \left(\frac{1}{\NGRID}\frac{V_\nsr^T}{V_\myo^T}\frac{d[\Ca]_\nsr}{dt}
  \right)+
  \sum \frac{1}{\NGRID} \frac{d[\CaBhtrpn]}{dt} + 
  \sum \frac{1}{\NGRID}\frac{d[\CaF]}{dt} = 0
\end{split}
\end{equation}
with $\NGRID=\frac{V_\myo^T}{V_\myo}=\frac{V_\nsr^T}{V_\nsr}$ is the number of
grid points.
We rewrite it either
\begin{enumerate}
  \item Option 1:
\begin{equation}
\begin{split}
 \sum \frac{d[\Ca]_\myo}{dt} + \NGRID  
  \sum\left(\frac{V_\ds}{V_\myo^T}\frac{d[\Ca]_\ds}{dt}\right) + 
  \NGRID \sum\left(\frac{V_\jsr}{V_\myo^T}\frac{d[\Ca]_\jsr}{dt}\right) + \\
  \sum \left(\frac{V_\nsr^T}{V_\myo^T}\frac{d[\Ca]_\nsr}{dt}
  \right)+
  \sum \frac{d[\CaBhtrpn]}{dt} + 
  \sum \frac{d[\CaF]}{dt} = 0
\end{split}
\end{equation}
and defining volume ratio (the same as the way we define volume ratio in
Sect.\ref{sec:balance-eq_non-spatial})
\begin{equation}
\lambda_\ds = \frac{V_\ds}{V_\myo^T};\;\;
\lambda_\jsr = \frac{V_\jsr}{V_\myo^T};\;\;
\lambda_\nsr = \frac{V_\nsr^T}{V_\myo^T}
\end{equation}

\item Option 2:
\begin{equation}
\begin{split}
 \sum \frac{d[\Ca]_\myo}{dt} +   
  \sum\left(\frac{V_\ds}{V_\myo}\frac{d[\Ca]_\ds}{dt}\right) + 
  \sum\left(\frac{V_\jsr}{V_\myo}\frac{d[\Ca]_\jsr}{dt}\right) + \\
  \sum \left(\frac{V_\nsr^T}{V_\myo^T}\frac{d[\Ca]_\nsr}{dt}
  \right)+
  \sum \frac{d[\CaBhtrpn]}{dt} + 
  \sum \frac{d[\CaF]}{dt} = 0
\end{split}
\end{equation}
and defining volume ratio (the same as the way we define volume ratio in
Sect.\ref{sec:balance-eq_non-spatial})
\begin{equation}
\lambda_\ds = \frac{V_\ds}{V_\myo};\;\;
\lambda_\jsr = \frac{V_\jsr}{V_\myo};\;\;
\lambda_\nsr = \frac{V_\nsr^T}{V_\myo^T} = \frac{V_\nsr}{V_\myo}
\end{equation}

\end{enumerate}

Given that all fluxes are defined based on VmyoT, then it need to be mapped to
Vmyo (which is the reference volume of Camyo). NOTE: \textcolor{red}{The below
fluxes are NOT affected by volume reference, Jncx, JCab, Jpmca, Jserca, Jtrpn,
JCaF}. Collectively, the first three fluxes are mapped to \verb!J_sarcomem!.
\begin{verbatim}
dCamyo = (J_efflux)_{if CRU} * VmyoT/Vmyo + J_leak * VmyoT/Vmyo + 
   J_sarcomem + J_ryr_nj_{if rogueRYR} * VmyoT/Vmyo - 
   J_serca - J_trpn - J_CaF + 
   Dmyox / dx^2 * (Camyo(i+1,j,k)-2*Camyo(i,j,k)+Camyo(i-1,j,k)) +
   Dmyoy / dy^2 * (Camyo(i,j+1,k)-2*Camyo(i,j,k)+Camyo(i,j-1,k)) +
   Dmyoz / dz^2 * (Camyo(i,j,k+1)-2*Camyo(i,j,k)+Camyo(i,j,k+1)) 
dCaBhtrpn = J_trpn   !flux forming compound CaB_htrpn in myoplasm
dCaF      = J_CaF    !flux forming compound CaF in myoplasm
\end{verbatim}
NOTE: To incorprate the effect of rogue RYR or peripheral RYR which sense the
[Ca]myo, \verb!J_efflux! and \verb!J_ryr_nj! are substituted by
only \verb!J_int2myo! in the calculation.

To achieve the fluxes balancing, they need to be mapped to the corresponding
volume
\begin{verbatim}
dCansr = Vmyo/Vnsr * (J_serca - (J_refill)_{if CRU} * VmyoT/Vmyo - 
          J_leak * VmyoT/Vmyo - J_ryr_nj_{if rogueRYR) * VmyoT/Vmyo + 
   Dnsrx / dx^2 * (Cansr(i+1,j,k)-2*Cansr(i,j,k)+Cansr(i-1,j,k)) +
   Dnsry / dy^2 * (Cansr(i,j+1,k)-2*Cansr(i,j,k)+Cansr(i,j-1,k)) +
   Dnsrz / dz^2 * (Cansr(i,j,k+1)-2*Cansr(i,j,k)+Cansr(i,j,k+1)) 
   
dCads = Vmyo/Vds * (J_ryr * VmyoT/Vmyo - J_efflux * VmyoT/Vmyo +
                   J_dhpr*VmyoT/Vmyo) 
dCajsr = Vmyo/Vjsr * (J_refill * VmyoT/Vmyo - J_ryr *
                VmyoT/Vmyo)

J_dhpr = I_dhpr * 1/(zCa*zF*Vol_ref)
Vol_ref = VmyoT
\end{verbatim}
NOTE: To incorprate the effect of rogue RYR or peripheral RYR which sense the
[Ca]nsr, \verb!J_refill! and \verb!J_ryr_nj! is substituted by \verb!J_nsr_lost!
in the calculation.


To incorporate the effect of fast buffering, we do
\begin{verbatim}
dCansr = beta_nsr * [ Vmyo/Vnsr * (J_serca - (J_refill)_{if CRU} * VmyoT/Vmyo - 
          J_leak * VmyoT/Vmyo - J_ryr_nj_{if rogueRYR) * VmyoT/Vmyo + 
   Dnsrx / dx^2 * (Cansr(i+1,j,k)-2*Cansr(i,j,k)+Cansr(i-1,j,k)) +
   Dnsry / dy^2 * (Cansr(i,j+1,k)-2*Cansr(i,j,k)+Cansr(i,j-1,k)) +
   Dnsrz / dz^2 * (Cansr(i,j,k+1)-2*Cansr(i,j,k)+Cansr(i,j,k+1)) 
                   ]
                   
dCads = beta_ds * [ Vmyo/Vds * (J_ryr * VmyoT/Vmyo - J_efflux * VmyoT/Vmyo +
                   J_dhpr)
                  ] 
dCajsr = beta_jsr * [ Vmyo/Vjsr * (J_refill * VmyoT/Vmyo - J_ryr *
                VmyoT/Vmyo)
                  ]
J_dhpr = I_dhpr * 1/(zCa*zF*Vol_ref)
Vol_ref = Vmyo
\end{verbatim}



\subsubsection{Referenced volume for the fluxes is Vmyo}

Assuming all grid points have the same cytosolic volume $V_\myo$.
\begin{equation*}
\begin{split}
 \sum \frac{d[\Ca]_\myo}{dt} +  
   \sum\left(\frac{V_\ds}{V_\myo}\frac{d[\Ca]_\ds}{dt}\right) + 
  \sum\left(\frac{V_\jsr}{V_\myo}\frac{d[\Ca]_\jsr}{dt}\right) + \\
  \sum \left(\frac{V_\nsr}{V_\myo}\frac{d[\Ca]_\nsr}{dt} \right)+ \sum
  \frac{d[\CaBhtrpn]}{dt} + \sum \frac{d[\CaF]}{dt} = 0
\end{split}
\end{equation*}

By defining volume ratio, notice that the cytosolic volume is for a single grid
point
\begin{equation}
\lambda_\ds = \frac{V_\ds}{V_\myo};\;\;\;
\lambda_\jsr = \frac{V_\jsr}{V_\myo};\;\;\;
\lambda_\nsr = \frac{V_\nsr}{V_\myo}
\end{equation}

Given that all fluxes are defined based on Vmyo (single grid-point), then 
\begin{verbatim}
dCamyo = (J_efflux)_{if CRU} + J_leak + J_sarcomem + J_ryr_nj_{if rogueRYR} -
   J_serca - J_trpn - J_CaF + 
   Dmyox / dx^2 * (Camyo(i+1,j,k)-2*Camyo(i,j,k)+Camyo(i-1,j,k)) +
   Dmyoy / dy^2 * (Camyo(i,j+1,k)-2*Camyo(i,j,k)+Camyo(i,j-1,k)) +
   Dmyoz / dz^2 * (Camyo(i,j,k+1)-2*Camyo(i,j,k)+Camyo(i,j,k+1)) 
J_trpn = dCaBhtrpn  !flux forming compound CaB_htrpn in myoplasm
J_CaF  = dCaF       !flux forming compound CaF in myoplasm
\end{verbatim}
with \verb!J_sarcomem! is described in Sect.\ref{sec:trans-sarc-flux}.
To achieve the fluxes balancing, they need to be mapped to the corresponding
volume
\begin{verbatim}
dCansr = Vmyo/Vnsr * (J_serca - (J_refill)_{if CRU} - J_leak - 
   J_ryr_nj_{if rogueRYR) + 
   Dnsrx / dx^2 * (Cansr(i+1,j,k)-2*Cansr(i,j,k)+Cansr(i-1,j,k)) +
   Dnsry / dy^2 * (Cansr(i,j+1,k)-2*Cansr(i,j,k)+Cansr(i,j-1,k)) +
   Dnsrz / dz^2 * (Cansr(i,j,k+1)-2*Cansr(i,j,k)+Cansr(i,j,k+1)) 
   
dCads = Vmyo/Vds * (J_ryr - J_efflux + J_dhpr)
dCajsr = Vmyo/Vjsr * (J_refill - J_ryr)

J_dhpr = I_dhpr * 1/(zCa*zF*Vol_ref)
Vol_ref = Vmyo
\end{verbatim}

To consider the fast buffering effect, finally, we have
\begin{verbatim}
dCamyo = beta_myo * [(J_efflux_{if CRU} + J_leak + J_sarcomem + 
           J_ryr_nj_{if rogueRYR} - J_serca - J_trpn - J_CaF) + 
       Dmyox / dx^2 * (Camyo(i+1,j,k)-2*Camyo(i,j,k)+Camyo(i-1,j,k)) + 
       Dmyoy / dy^2 * (Camyo(i,j+1,k)-2*Camyo(i,j,k)+Camyo(i,j-1,k)) + 
       Dmyoz / dz^2 * (Camyo(i,j,k+1)-2*Camyo(i,j,k)+Camyo(i,j,k+1)) 
                    ]
   
dCansr = beta_nsr * [ (J_serca - J_refill_{if CRU} - J_leak -
             J_ryr_nj_{if rogueRYR})/lambda_nsr + 
       Dnsrx / dx^2 * (Cansr(i+1,j,k)-2*Cansr(i,j,k)+Cansr(i-1,j,k)) +
       Dnsry / dy^2 * (Cansr(i,j+1,k)-2*Cansr(i,j,k)+Cansr(i,j-1,k)) +
       Dnsrz / dz^2 * (Cansr(i,j,k+1)-2*Cansr(i,j,k)+Cansr(i,j,k+1)) 
                   ]
   
dCads = beta_ds/lambda_ds * (J_ryr - J_efflux)
dCajsr = beta_jsr/lambda_jsr * (J_refill - J_ryr)
\end{verbatim}
with \verb!beta_ds! is assumed to be calculated the same way as \verb!beta_myo!.
To assume there is no loss in calcium at boundary due to diffusion, we always
have Camyo(0) = Camyo(1), and Camyo(n) = Camyo(n+1).

Other Calcium-bound complexes:
\begin{verbatim}
dCaBhtrpn = kon_htrpn*Ca_myo*(B_htrpnT - CaB_htrpn) - koff_htrpn*CaB_htrpn  
dCaF = kon_F * (Ca_myo_in(current) * F_in(current)) - koff_F * CaF_in(current) 
\end{verbatim}

\section{Parameters in a model}
\label{sec:parameters-model}

\begin{verbatim}
zCa = 2         (valence)                               [unitless]
zNa = 1
zK  = 1   
\end{verbatim}

\begin{verbatim}
time            (current simulation time)               (sec)
t_start         (when start stimulation)                (sec)
t_end           (when stimulation end)
write_firstbeat    (write out FB or not?)  0 or 1
duration_write_lastbeats    duration to write-out (counting from the end)
write_restartfile    write to restart-file?       0 or 1

NSFU            (number of CRUs)                        [unitless]
                20,000
N_L             (number of LCC in a CRU)
                7
N_R             (number of RyR in a CRU)
                49 
% mL              (number of Markovian state single LCC)
%                 6
% mR              (number of Markovian state single RyR)
%                 2

## Voltage (membrane potential)
Vm              (membrane potential)                    (mV)

V_rest          (resting membrane potential)            (mV)
                -8.5d1 mV
V_stim          (stimulated potential)
                -1.0d1, 1.0d1, ... mV
I_stim          (stimulated current density)            (uA/cm^2)
                -6.5189 (uA/cm^2) ~ -1 (nA) with Am = 1.534d-4 cm^2

dt_LB           (lower-bound time step)                 (sec)
                1d-8 sec
dt_UB           (upper-bound time step)
                1d-6 sec

t_V_stim_start  (when to start V-stimulation)           (sec)
                10ms ~ 1d-2 s
t_V_stim_duration  (duration of V-stimulation) 
                100ms ~ 1d-1 s
t_I_stim_start
                10ms 
t_I_stim_duration
                5ms ~ 5d-3 s

Hz              (frequency of stimulation)              [Hz] = (1/sec)
                number of stimulations per sec
%write_interval  (how often to write data out)           (ms)
%                0.1 ms
write_interval  (how often to write data out)           (s)
                0.1d-3 s

### ionic concentrations
Na_o            (out-cell sodium concentration)         (uM)
                1.4d5 uM  = 140 mM
Na_i                                                    (uM)
                1.02d4 uM = 10.2mM
K_o                                                     (uM)
                5.4d3 uM  = 5.4 mM
K_i                                                     (uM)
                1.43720d5 uM = 143.7205 mM
Ca_o                                                    (uM)
                1.8d3 uM  = 1.8 mM
Ca_myo                                                  (uM)
                old: 9.00000d-02 uM (0.09 uM)
                new: 1.0d-1 uM (0.1 uM)
Ca_nsr                                                  (uM)
                1.00000d+03 uM (~ 1mM)

Ca_ds           (calcium in subspace)                   (uM)
                = Ca_myo (initially)
Ca_jsr          (calcium in jsr)                        (uM)
                = Ca_nsr (initially)

Csc             (specific membrance capacitance)        [uF/cm^2]
                1.0d0
#NOTE: Csa = capacitance of the cell's surface area [pF]

zF              (Faraday constant)                      [C/mol]  = [mJ/(mV.mol)]
                9.6485d4
zR              (universal gas constant)                [mJ/(K.mol)]
                8.314d3
zT              (temperature)                           [K]
                310K

eta_RyR         (cooperativity number - calcium bind to RYR) [unitless]
                2.2d0
                unused: 2.3d0 (stabilizing open ryr better)
                
eta_LCC         (...- bind to LCC)              [unitless] 
                2.00000d0
                
eta_Ca_CaM      (... - calcium bind to Calmodulin)
				2.0
				                
eta_pmca        (...of plasma membrane Ca2+-ATPase)     [unitless]
                2.0d0

Ej              (average RyR allosteric connection)     [unitless]
                =1/(N_R*(N_R-1)) *
                 ((sqrt(N_R)-2)^2*4+(sqrt(N_R)-2)*4*3+4*2)
## For N_R = 49, Ej = 0.0714

Ej_small        (... for small cluster)                 [unitless]

Ecc             (allosteric coupling energy...)         [unitless]
                when RyR and its neighbor are both at 
                Close (C) state
                -9.2
Eoo                                                     [unitless]
                when RyR and its neighbor are both at 
                Open (O) state
                -8.5
                
## NOTE: we already remove the energy term k_BT from the formula of 
#  allosteric coupling, with  1 [k_B.T] = 1.381x10^{20} [mJ]
#  with k_B = Boltzmann constant
#       T   = temperature [K]
# Ecc and Eoo depends on mR, the number of state of 
# single RyR channels
# mR = 2 : Ecc = -0.92, Eoo = -0.85

## volume of cell components
V_cell          (cell volume)                           (pL) 
                36.8 pL

## values in parameters files should be in (pL) 
V_myo_T         (volume myoplasm)                       (pL)
                + ventricular myocyte
                rat: 1.8d1 pL

V_nsr_T
                + ventricular myocyte
                rat: 1.08d0 pL
V_jsr_T
                + ventricular myocyte
                rat: 9.0d-2 pL
V_ds_T
                + ventricular myocyte
                rat: 2.7d-2 pL

### Volume of cell components in spatial model
cell_x_len = 100 um
cell_y_len = 20  um
cell_z_len = 18  um
## V_cell = 36,000 um^3 = 36x10^3x(10^-5)^3x10^12 pL = 36pL 
# NOTE: 1pL = 1000 um^3



## Ca2+ buffering fraction [from 0..1], i.e. fraction of Ca2+ is free
beta_ds         (in subspace)                           [unitless]
                fixed: 0.1d0, i.e. 90% is bound, and 10% is free
beta_nsr           (in network SR)
                fixed: 1, i.e. 100% is free (i.e. no buffer)

## total Ca2+ buffer concentration
# 2 main buffers in the myo: Troponin and Calmodulin (CaM)
B_myoT          (CaM-liked buffer in myoplasm)          (uM)
                143 uM 
B_jsrT          (in JSR)
                14000 uM
B_htrpnT        (in myoplasm)
                1.40000d+02

## Assume fast buffering for B_jsr and B_myo
# The half-saturation constants are
Km_myo                                                  (uM)
                9.60000d-1
Km_jsr                                                  (uM)
                6.38000d+02

## For slow-buffering, the reaction constants are
kon_htrpn       troponin Ca on rate                     [1/(uM*s)]
                2.37000d+00
koff_htrpn      troponin Ca off rate                    (1/sec)
                3.20000d-02

## transfer rate of Ca2+ (defined based on 20,000 CRUs and V_myo_T)
v_refill_T      summation from 20,000 CRUs              (1/sec)
                5.0d0 (rat)
v_efflux_T
                2.5d2 
v_leak          (transfer rate across NSR to bulk myo)  (1/sec)
                0d0
v_ryr_T
                old: 4.72419d1
                new: 5.64279d1 
# NOTE: If V_myo_T is reduced half, we also need to reduce CRUs half
#       for these rates to be unchanged
# However, in non-spatial simulation, we often run with a smaller
# amount of CRUs, while keeping the V_myo_T unchanged. That's why 
# we need to increase the per channel rate using
# v_ryr = v_ryr_T / (NSFU * N_R)
# In spatial simulation, if we work with myoplasmic volume of a 
# single grid point, we don't have to change v_ryr.

# percent of non-junctional RyR compared to junctional RyR
prcent_ryr_nj   (percent)                               [unitless]
                5d-2

## maximum total permeability (at whole-cell level)
# with DHPR are locate at CRUs 
P_dhpr_T                              [10^-3cm^3/sec] or [10^-6 L/sec]
                1.8721d-4   (when i_1dhpr = 0.25 nA)
                other to try: 1.1233d-3 (when i_1dhpr = 0.15 nA) - not work
# NOTE: At whole-cell level, we use [L/sec], rather than (uM/sec)
# for the permeability

# percent of DHPR are located at sarcolemmal membrane, but not in CRU
# compared to junctional DHPR
prcent_dhpr_nj  (percent)                               [unitless]
                10d-2


## SERCA-pump
Ap              (concentration of SERCA molecules)      (uM)
                150 uM
# assuming Ca-binding to SERCA is fast                
Kp_myo          (dissociation constant of Ca-binding    (uM)
                to SERCA on myoplasmic side)
                0.91d+3 uM             
Kp_nsr          (dissociation constant of Ca-binding    (uM)
                to SERCA on NSR side)
                2.24d+3 uM

Am              (whole-cell membrane area)              (cm^2)
                1.5340d-4 (cm^2)
Acap = Am*Csc   (capacitative membrane area of cell)    [uF]


## Maximum conductance of different ion channels
# We preferably use [mS/cm^2] rather than [ms.uF]
g_bCa           (background Ca2+ conductance)           [mS/cm^2] 
                old: 7.36351d-4 mS/cm^2 (work with cmyo = 0.09 uM)
                new: 8.6918d-4 mS/cm^2  (work with cmyo = 0.1 uM)
g_Na
                1.3d1 mS/cm^2
g_K1
                0.20  mS/cm^2
g_Kss
                2.105d-2 mS/cm^2
                other to try: 4.12d02
g_Ktof
                0.0798d0 mS/cm^2 
g_Ktos
                3.145d-2 mS/cm^2
                other to try: 6.29d-2                
g_bNa
                1.066d-4 mS/cm^2
g_bK
                0.0d0 mS/cm^2

## Maximum Ca2+ current density
I_pmca_bar      (via plasma-membrane Ca-ATPase)         (uA/cm^2)
                0.75d0 
I_ncx_bar       (via Na/Ca-exchanger)                   (uA/cm^2)
                1.0d3

# Maximum Na+ current density
I_NaK_bar       (via Na/K-exchanger)                    (uA/cm^2)
                0.88d0

#NOTE: J (flux) = I x ... [mol/(cm^2.sec)]
#      are defined based on V_myo_T

% ## Rate transitions ([uM^eta.sec^{-1}] or [sec^{-1}] depending 
% #    on the scheme)
% # with eta is the Hill coefficient
% kR12                        ! C-> O [1/(s.uM^-eta_RyR)]
% kR21                        ! O-> C (1/sec)
% kL25
% kL16
% kL34
% kL21
% kL56
% kL23
% kL54
% kL52
% kL12
% kL32
% kL65
% kL45   (remove all transition rates)

k_jsr0                                                  (uM^-1)
         2.30000d-04   
k_jsr1                                                  [unitless]
         2.00000d-02   

v_theta_dhpr                                            (mV)
 	     2.0d0
sigma_v_dhpr                                            (mV)
         6.50d0
         
### read-in but not being used for now, may be we need to remove????
V_theta2, V_sigma2


# cell size (um)
1.00d2        x_length
2.00d1        y_length
1.80d1        z_length

## Grid element dimension (0.1 um or 0.2 um)
dX_                                                     (uM)
dY_
dZ_   


## Diffusion constants of free Ca2+
# in myoplasm
Dmyo_x                                                  [um^2/sec]
           note: use 225 um^2/sec 
           other to try: 300 um^2/sec
Dmyo_y
Dmyo_z   
# in NSR
Dnsr_x                                                  [um^2/sec]
           use 10 um^2/sec
Dnsr_y
Dnsr_z

lspacing    long_spacing_CRU                            (uM)
           2.0 or 1.6
tspacing    tran_spacing_CRU
           0.8 or 0.6 
                                 
\end{verbatim}

% boundary_condition         
%            1.0 = Dirichlet
%            2.0 = Neumann
%            3.0 = Robin

\section{Computed variables}
\label{sec:computed-variables}

\begin{verbatim}
## Some computed quantities
RT_F = RT/F                                             [mJ/C] = (mV)
              26.7123387
F_RT = F/RT                                             [1/mV]
              0.0374358835
RT_FzCa = RT_F/zCa                                      (mV)
              13.3561693528

# use in calcium current calculation
Am_2FVmyo     (old name: Am2FVmyo)                      [mol.cm^2/(C.L)]
              = Am*1d12/(zCa*zF*V_myo_T)
              30.76401 [mol.cm^2/(C.L)]

# use in Na, K current calculation
Am_FVmyo                                               [mol.cm^2/(C.L)]
              = Am*1d12/(zF*V_myo_T)

#may be used by EC-coupling model
F2zCa2_RT=zF**2*zCa**2/(zR*zT)                          [C/(mol.mV)]
              =14448.004878
FzCa_RT=zF*zCa/(zR*zT) = 1/RT_FzCa                      [1/mV]
              =0.0748717670153

inv_2FVmyo = 1.0d12/(zCa*zF*V_myo_T)                    [mol/(C.L)]

## Volume
V_ds          (volume of a single subspace)            (pL)
               = V_ds_T/dfloat(NSFU)
V_jsr         (volume of a single JSR)
               = V_jsr_T/dfloat(NSFU)
## NOTE: In non-spatial simulation, we often run with a smaller amount
#   of CRUs, while keeping V_myo_T unchanged. That's why we need to 
#   increase the volume occupied by a single CRU.
#   In spatial simulation, V_ds, V_jsr should be set fixed 
#  (see spatial model)

## Volume fraction (which is being used to map 
#  fluxes with concentration defined over V_myo_T
#  to the concentration defined on the corresponding volume
#  when we find dc/dt
lambda_ds, lambda_jsr, lambda_nsr                    [unitless]
    lambda_ds = V_ds/V_myo_T
    lambda_jsr = V_jsr/V_myo_T
    lambda_nsr = V_nsr_T/V_myo_T
# to avoid computational drop, we use inv_lambda
inv_lambda_ds = V_myo_T/V_ds                         [unitless]
inv_lambda_jsr = V_myo_T/V_jsr
inv_lambda_nsr = V_myo_T/V_nsr_T


## may be used by spatial whole-cell model
invKp_myo = 1d0/Kp_myo
invKp_nsr = 1d0/Kp_nsr
invzCa = 1.0d0/zCa                                   [unitless]

dx2 = dX_^2                                          [um^2]
dy2 = dY_^2                                          [um^2]
dz2 = dZ_^2                                          [um^2]

MAXX, MAXY, MAXZ                                     [unitless]
\end{verbatim}


\section{Transfer rate}
\label{sec:transfer_rate}


A transfer rate is defined by the transport of number of molecules $n$ per an
ion channel per second (or number of moles $(n/N_A)$per an ion channel per
second, with $N_A$ is the Avogradro number. Ions move faster through ion
channels than through carriers (exchangers)
\begin{enumerate}
  \item throughput through ion channels: $10^6-10^8$ ions/sec, which maps to
  current equivalent of $10^{-12}-10^{-10}$ Amperes (or 1-100pA), assuming the
  ion is monovalent. 
  \item transfer rate for Na/K-ATPase is 300 $\Na$ \& 200$\K$ /sec, which maps
  to current equivalent 1.5$\times 10^{-17}$Amperes. 
\end{enumerate}

The transfer rate is dependent upon the concentration, but not the volume. So,
there's no change in the rate when shifting to spatial model at all.
% So,
% However, in our model, the transfer
% rate is defined in unit of (1/sec). The reason is that they are mapped to total
% myoplasmic volume ($V_\refe = V_\myoT$ (pL)). It means that if the transfer rate
% is A (pL/s); then the value that we keep is $\frac{A}{V_\myoT}$ (1/s). 
% In the spatial model, the transfer rate is mapped to the myoplasmic volume of a
% single grid point $V_\refe = V_\myo$. 

\subsection{Non-spatial ($V_\myoT$)}
\label{sec:transfer_rate-non_spatial}

This requires the same convention in other quantities, i.e. calcium
concentration should be defined on this $V_\myoT$, $v_\rf, v_\ef, v_\ryr$ should
be defined over $V_\myoT$. If we use a different volume reference, we need to
convert these values accordingly. It means that if we have $v_\rf^T=5.0d0$
(1/s) defined for 20,000 CRUs on $V_\myoT=1.8$pL. Then, the efflux for a single
CRU defined over $V_\myoT=1.8$pL is 
\begin{verbatim}
v_efflux = v_efflux_T / 20000
\end{verbatim}
Now, the efflux for a single CRU defined over $V_\myoT=a$ (pL) then
\begin{verbatim}
v_efflux_a = v_efflux * V_myo_T/a
\end{verbatim}

Example: The transfer rate via a single RyR channel
\begin{verbatim}
v_ryr      (the maximum transfer rate for a single channel)         (1/sec)
           = i_1ryr*1.0e6/(1.0e3*zCa*zF*V_myo_T)
\end{verbatim}

Here, \verb!v_ryr! should be explicitly defined as
\begin{verbatim}
    v_ryr = i_1ryr * 1e6/ (DCa * zCa * F * V_myo_T)
# with i_1ryr = 0.2pA (single RyR current)
#      DCa =  [Ca]jsr - [Ca]ds ~ 1000 uM
#      zCa = 2
#      F   = Faraday constant (C/mol)
#      V_myo_T = volume myoplasmic total (pL)
DCa = Ca_nsr-Ca_ds ~ 1000 (uM)
zCa = 2
zF  = 9.6485d4 [C/mol]
i_1ryr                                                      (pA)
         = 0.2 (pA)
V_myo_T                                                    (pL)
         = 18 pL (rat cardiac cell)
         
v_ryr_T    (the transfer rates at whole-cell level)         (1/sec)
           = v_ryr * NSFU * N_R
\end{verbatim}
So (defined based on $V_\myoT$), 
\begin{enumerate}  
  \item \verb!i_1ryr = 0.22! (pA) then \verb!v_ryr_T! = 62.07688 (1/s)
  \item \verb!i_1ryr = 0.2! (pA) then \verb!v_ryr_T! = 56.4279 (1/s)
  \item \verb!i_1ryr = 0.17! (pA) then \verb!v_ryr_T! = 47.9637 (1/s)
  \item \verb!i_1ryr = 0.18! (pA) then \verb!v_ryr_T! = 50.78 (1/s)
  \item \verb!i_1ryr = 0.185! (pA) then \verb!v_ryr_T! = 52.19 (1/s)
  \item \verb!i_1ryr = 0.15! (pA) then \verb!v_ryr_T! = 42.3209134 (1/s)11
\end{enumerate}

The flux via RyR is defined as in eq.~\eqref{eq:19}. So, for a single
channel, the flux is defined as
\begin{equation}
  \label{eq:44}
      J_{1\ryr} = v_\ryr.([\Ca]_\jsr^\ii-[\Ca]_\ds^\ii)
\end{equation}
On the other hand, the flux is defined via the single channel current
$i_{1\ryr}$ as
\begin{equation}
  \label{eq:45}
  J_{1\ryr} = i_{1\ryr}\frac{1}{z_\ca F V_\myo^T}
\end{equation}
So, the transfer rate is found by the equation
\begin{equation}
  \label{eq:46}
  v_\ryr = \frac{i_{1\ryr}}{z_\ca F V_\myo^T([\Ca]_\jsr - [\Ca]_\ds)}
\end{equation}

Now:
\begin{enumerate}
  \item  applying $[\Ca]_\jsr = 1e3\mu$M, $[\Ca]_\ds = 0.1\mu$M, $z_\ca=2$,
$F=9.6485e4$ (C/mol), and $V_\myo^T=25.84\times 10^{-6}\mu$L,
$i_{1\ryr}=0.2$(pA)
\begin{equation}
  \label{eq:47}
  v_\ryr = 4.01095407602\times 10^{-5}  \;\;\;\text{[1/s]}
\end{equation}
and 
\begin{equation}
  \label{eq:48}
  v_\ryr^T = v_\ryr\times 20000\times 49 = 3.9307349945d1 \;\;\text{[1/s]}
\end{equation}
  With this rate, and using the smaller VmyoT=18.4pL, then the single channel
  current is only 0.14pA.
  
  \item If we use the smaller volume VmyoT=18.4pL, to achive the same single
  current, we need a larger single current transfer rate. If $i_{1\ryr}=0.2$(pA)
  %applying $[\Ca]_\jsr = 1e3\mu$M, $[\Ca]_\ds = 0.1\mu$M, $z_\ca=2$,
%$F=9.6485e4$ (C/mol), and $V_\myo^T=18.4\times 10^{-6}\mu$L. To 
\begin{equation}
  \label{eq:47}
  v_\ryr = 5.63277\times 10^{-5}  \;\;\;\text{[1/s]}
\end{equation}

\end{enumerate}

The refill rate from NSR to JSR (defined based on $V_\myoT$)
\begin{equation}
v_\rf = v_\rf^T/NSFU
\end{equation}

References:
\begin{itemize}
  \item \url{http://www.convert-me.com/en/convert/flow_rate_volume}
\end{itemize}

%\subsection{Spatial ($V_\myo$)}
\subsection{Spatial}
\label{sec:transfer_rate-spatial}

The transfer rate doesn't change upon the change of volume, but the change of
concentration. So, even the flow from subspace to bulk myoplasm in the
compartmental model turns into the flow from subspace to a single grid point,
the rate is the same, as the concentration of the single grid point is like the
one of bulk myoplasm. Even though, there can be some difference which requires
the adjustment of the parameters. But there's no significance change in the
value.

IMPORTANT: The volume being used is considered as $V_myo_T=18.4$ (pL).
% 
% The transfer rate is defined based on the volume of a single grid point $V_\myo$
% with the dimension is 0.2x0.2x0.2 ($\mu$m$^2$). 
% \begin{equation}
%  v_\ryr = \frac{i_{1\ryr}}{z_\ca F V_\myo ([\Ca]_\jsr - [\Ca]_\ds)}
% \end{equation}
% and
% \begin{equation}
% v_\rf = \frac{v_\rf^T}{20000} *
% V_{\myo(true)}^T/\text{Vol\_ref}
% \end{equation}
% with \verb!V_myo_T_true! is real-size cell total myoplasmic volume,
% \verb!Vol_ref! is the reference volume which is equal to $V_\myo$ in our case. 
% 
% Given the diffusion constant $D_\myo=250\mu$m$^2$/s, and the distance about
% dx=0.05-0.1$\mu$m$^2$, the transfer rate can be estimated to be $D_\myo/(dx)^2$,
% or $v_\ef \sim 25,000-100,000$ (1/s).
% 
% \begin{verbatim}
% v_leak = v_leak * V_myo_T_true/Vol_ref
% \end{verbatim}
% However, for simplicity, we just use \verb!v_leak=0.0d0!.

\section{Channel current}
\label{sec:channel_current}

The current is the net flow of charge across the membrane via channel pore. The
driving force to push the ion across the membrane via channel pore is dependent
on the concentration gradient and the voltage.

The direction of the current flow is the direction of the flow of positive
charge relative to the inside of the cell
\begin{enumerate}
  \item Inward current: positive charge flowing into the cell, i.e. negative
  charge flowing out of the cell, e.g. LCC channel
  (Sect.\ref{sec:LCC_permeability})
  \item Outward current: positive charge flowing out of the cell, e.g. 
\end{enumerate}

Single channel current is calculated using Ohm's law: $i=g(V_m-E_\rev)$. 
\begin{enumerate}
  \item $i$ (pA) single channel current
  \item $g$ (pS) single channel conductance
  \item $V_m$ (mV)
\end{enumerate}

For macroscopic current, we use the single channel current and the total number
of channel $n$ and the opening probability $P_o$
\begin{equation}
I = i\times n \times P_o
\end{equation}
with $I$ ($\muA$). 
\begin{enumerate}
  \item If the channel is non-voltage gated, the single channel i/V curve is
  linear (Ohmic) and so the macroscopic I/V curve but with greater slope.
   
  \item If the channel is voltage-gated, the $P_o$ plot is sigmoidal
of $V_m$, i.e. increasing voltage increase $P_o$ to a point and then saturate.
So, still i/V curve is linear, but the macroscopic I/V curve has both
exponential and linear components. 
\end{enumerate}
NOTE: I/V curve tells us nothing about the gating. It only tells us what is the
current when the channel open.	

\section{Rectifying $\K$ channel}

A rectifying channel is a voltage-gated channel (Sect.\ref{sec:channel_current})
but passes current better in one direction that it does in another. 

\begin{enumerate}
  \item $\K$ inward rectifier: the channel open at rest, and allow ions to flow
  into the cell, but not out. 
\end{enumerate}


\section{LCC Permeability}
\label{sec:LCC_permeability}

\subsection{Single channel}
\label{sec:permeab-single_channel}

The permeability of a single ion channel is defined as
\begin{equation}
  \label{eq:41}
  P_\dhpr = \frac{i_\dhpr}{z_\ca ^2 \frac{F}{RT/F}V_m
      \left[\frac{\exp\left(\frac{z_\ca V_m}{RT/F}\right)[\ca]_\myo-0.341[\ca]_o}{\exp\left(\frac{z_\ca V_m}{RT/F}\right)
          - 1} \right]} 
\end{equation}
with $i_\dhpr=0.25$pA (OLD) and  is the single channel current measured at
$V_m=-10$mV, with $[\Ca]_\myo=0.1\mu$M, and
$[\Ca]_o=2$mM=$2\times10^3\mu$M, which gives
\begin{equation}
  \label{eq:43}
  P_\dhpr = 1.337245085\times 10^{-15} \;\; \text{L/s}
\end{equation}
and
\begin{equation}
  \label{eq:42}
  \begin{split}
      P_\dhpr^T=P_\dhpr\times 20,000 \times 7 &= \\
      % 0.187214311\times 10^{-6} \;\;\text{cm$^3$/s} \\
      % &= 1.87214311\times 10^{-3} \;\;\text{$\mu$m.cm$^2$/s}
  &= 1.87214311\times 10^{-10} \;\;\text{L/s}
  \end{split}
\end{equation}

In the spatial model, the function \verb!find_Pdhpr_singlechannel()! will return
the permeability for a single channel. 

% \begin{verbatim}
% Vm = -10d0
% Ca_o = 2d3 
% P_dhpr = i_1dhpr / ((4*zF*Vm)/RTF * ((0.1d0*exp(2*Vm/RTF)-0.341 *
%          Ca_o)/(exp(2*Vm/RTF)-1))) 
% NSFU = 20000
% N_L = 7
% P_dhpr_T = P_dhpr * NSFU * N_L
% 
% # In spatial simulation, when we reduce V_myo_T, we also need to
% # reduce NSFU as well, so we don't have to change P_dhpr, and it must
% be defined using  
% # P_dhpr = P_dhpr_T/ (20000 * N_L)
% 
% # v_efflux and v_refill doesn't change????? [TUAN]
% 
% \end{verbatim}
For the case of non-junctional LCC, there is a number of choice 
\begin{enumerate}
  \item we can assume its contribution is equally
distributed to every grid element, so the corresponding
permeability contributed to each grid point is
\verb!P_dhpr_nj/NUM_GRID_ELEMENTS!. 
  \item  model them as a number of small cluster neighboring to the large
  cluster. This will be published in a separate paper. 
\end{enumerate}

\subsection{Whole-cell level}
\label{sec:permeab_whole-cell}

The permeability at whole-cell level is defined as
\begin{verbatim}
P_dhpr_T  = P_dhpr * (20000 * N_L) 
\end{verbatim}

In the simulation, we can use smaller than 20,000 release sites. In this case,
we can assume that the flux from a single release is larger than usual, by
dividing NSFU.
\begin{verbatim}
P_dhpr  = P_dhpr_T / (NSFU * N_L) 
\end{verbatim}
\verb!P_dhpr_nj! in non-spatial model is based on whole-cell. If we model
non-junctional LCC as 10\% of whole-cell LCC in the junctions, then the
permeability is \verb!P_dhpr_nj = P_dhpr_T * 10%!.

\section{RYR permeability}	



\section{Bacground conductance}
\label{sec:bacground_conductance}

The background conductance is chosen so that it cancel out the change of calcium
concentrations in the myo during quiescent stage of the cell. During this stage,
the two available calcium currents are $I_\ncx$ and $I_\pmca$. So, $g_\bCa$ is
chosen so that $J_\bCa+J_pmca+J_\ncx = 0$.
 

\section{dp\_arg(1\ldots6)}
\label{sec:dp_arg}

These are parameters passed to the kernel to avoid uncessary evaluations. 
\begin{enumerate}
  \item Leak model: \verb!dp_arg1, dp_arg2!
  \item ECC model: \verb!dp_arg1, dp_arg2!
  \begin{verbatim}
dp_arg1 = EXP(Vm*FzCa_RT)
dp_arg2 = 1.d0 / (dp_arg1 - 1.d0)
  \end{verbatim}
  
  \item Spatial model: (assume Vm is homogeneous) \verb!dp_arg1, dp_arg2!,
  \verb!dp_arg3, dp_arg4, dp_arg5, dp_arg6!
  \begin{verbatim}
dp_arg3 = EXP(eta_ncx*Vm*F_RT)
dp_arg4 = EXP((eta_ncx-1.d0)*Vm*F_RT)
dp_arg5 = 1 / ((Km_ncxNa**3 + Na_o**3)*(Km_ncxCa+Ca_o)*(1.d0+ksat_ncx*dp_arg4))
dp_arg6 = Na_o**3
  \end{verbatim}
\end{enumerate}

\section{Concentration}
\label{sec:concentration}

In chemistry, a molar concentration is defined as the amount of a substance (in
mole) within a given volume. In biology, molar concentration is used with unit
of [$\mu$M] or micromole per litre. 
\begin{itemize}
  \item $[\Ca]_\ds$ is the amount of free calcium within the subspace (defined
  over the volume of subspace $V_\ds$)
  \item $[\Ca]_\jsr$ is the amount of free calcium within the junctional-SR
  (defined over the volume of $V_\jsr$)
  \item $[\Ca]_\nsr$ is the amount of free calcium within the network-SR
  (defined over the volume of $V^T_\nsr$)
  \item $[\Ca]_\myo$ is the amount of free calcium within the myoplasm (defined
  over the volume of $V^T_\myo$)
\end{itemize}

In the spatial model, for a single grid point, the concentration is defined as
\begin{itemize}
  \item $[\Ca]_\nsr$ is the amount of free calcium within the network-SR
  of a single grid point (defined over the volume of $V_\nsr$)
  \item $[\Ca]_\myo$ is the amount of free calcium within the myoplasm
  of a single grid point (defined over the volume of $V_\myo$)
\end{itemize}

Total free SR calcium concentration is 
\begin{verbatim}
Ca_SRT = (Ca_NSR * VnsrT + Cajsr_T * Vjsr) / (VnsrT + VjsrT)
\end{verbatim}



\section{Range of values}
\label{sec:range_values}

\subsection{Non-spatial}
\label{sec:ranges_non-spatial}

Typically, the range is varied within 20-30\% of the mean of the value. Here are
some information 
\begin{enumerate}
  \item \verb!v_efflux_T! : a freedom parameter, typically, we choose so
  that it keeps the peak within 100-150$\mu$M. In non-spatial model, we use
  \verb!v_efflux_T=250! (1/s) defined based on $V_\myoT$.
  
  \item \verb!v_refill_T! : a value depending on the size of $V_\jsr$.
  In the non-spatial model, we choose 5.0d0 (1/s) defined based on $V_\myoT$.
  
  \item \verb!v_leak_T! : we set to zero assuming there is no leak, other than
  the 3 proposed mechanisms: (1) $\Ca$ sparks, (2) invisible $\Ca$ leak, and
  (3) leak via non-junctional RyRs.
  
  \item \verb!v_ryr_T!: it's dependent upon the single channel RyR channel (see
  Sect.\ref{sec:transfer_rate})
  
  \item \verb!P_dhpr_T!: the peremeability (L/sec) is dependent upon the single
  channel current (see Sect.\ref{sec:LCC_permeability}).
\end{enumerate}

\subsection{Spatial}
\label{sec:ranges_spatial}

In the spatial, the ranges are defined based on a reference volume, which in our
case is the $V_\myo$ of a single grid point.

\section{What we can do to test}
\label{sec:what-we-can}

\begin{enumerate}
\item Change $i_\ryr$ to see how the sparks and $\Ca$
  waves/oscillations in cell
\item Change $i_\dhpr$ to see how the sparks and $\Ca$
  waves/oscillations and AP duration (APD) in cell

\item Change $[\Ca]_o$ and $[\Ca]_\nsr$ to see how the ....

\item Do I-clamp (1nA for 5ms) to see AP with simulation long enough
  (e.g. 10s). The frequency is 1Hz.

\item Do V-clamp with different values ranging from -40mV to +40mV,
  and run long enough. The frequency is 1Hz.

\item emulate caffeine-dump effect (Sect.\ref{sec:caffeine-dump}) 
\item emulate effect of ruthenium red by
  \begin{itemize}
  \item inhibit all CaRUs at a certain time period: 
  \end{itemize}
\end{enumerate}


Search for 'TUAN PLAN' to see future change.

\subsection{Trigger a spark}
\label{sec:spark_trigger}


Typically, we want to trigger a spark to study what's its effect on the
neighborhing sites. A spark is triggered when all RyRs in the cluster open.
To do so, we can
\begin{enumerate}
  \item Hold one LCC channel open for 0.5ms or 1ms (whaterver it takes to ensure
  a spark). \citep{sobie2002tcas} opened a stereotypical DHPR opening (0.5pA,
  0.5ms) which can trigger a spark with fidelity $> 98\%$ \citep{sobie2002tcas}.
  \begin{verbatim}
  trigger_CRU_openDHPR(...)
  \end{verbatim}
  
  \item Open more than 6 or 8 RyR channels at the same time: this pretty much
  guarantee a spark. Two different names of the same function
  \begin{verbatim}
CALL trigger_CRU_openRYR(U_Ro, CRU_states(1:NSFU), NSFU, irow_R, &                                                                                                                                                   
            LCCcluster_states, RYRcluster_states, is_all_CRU_open)    
!or

trigger_CRU_2
  \end{verbatim}
  This can be the effect of applying 15mM caffeine for 400ms \citep{song1997}.
  Previously, we use the name \verb!trigger_CRU()!, yet to be clearer, we rename
  to the name \verb!trigger_CRU_openRYR()!.
  
  \item Open RyR channel one by one, after a certain time period or a number of
  time steps (in the case of adaptive time-step). The number of time-steps is given
  by \verb!NUM_TIMESTEP_OPENCHANNEL!. Then, we use the code
  \begin{verbatim}
  ! how many time step before we open a new channel (KEEP it long enough)
INTEGER, PARAMETER :: NUM_TIMESTEP_OPENCHANNEL = 2 
INTEGER :: time_step_count_openchannel
  
IF (time_step_count_openchannel .EQ. NUM_TIMESTEP_OPENCHANNEL) THEN 
   sfu_state = sfu_state_dev
   CALL trigger_CRU_1(U_Ro, sfu_state, NSFU, is_all_CRU_open)
   sfu_state_dev = sfu_state
   RYRgate_dev(1:NSFU) = U_Ro(sfu_state(1:NSFU))
   DHPRgate_dev(1:NSFU) = U_Lo(sfu_state(1:NSFU))
   RYRgate_dev(NSFU+1:) = U_Ro_small(sfu_state(NSFU+1:))
   time_step_count_openchannel = 1
ELSE
   time_step_count_openchannel = time_step_count_openchannel + 1
ENDIF  
  \end{verbatim}
  
\end{enumerate}


\section{Nifedipine}
\label{sec:nifedipine}

Nifedipine (1$\mu$mol/L) can be used to reduce $I_\ca$ current, without changing
single channel current amplitude, but change mean open time~\citep{santana1996}.
It reduces number of evoked sparks. 

\section{About mean/std}
\label{sec:about-meanstd}

Suppose you have a dataset and you have calculated the standard
deviation (std) or variance, but don't have the mean. If you have some
new data to include in the dataset. Can we estimate the new std
without knowledge of the mean? - The answer is no.

Suppose you have a complete dataset, can we estimate the std or
variance without calculating the mean first? The answer is yes, as
long as we can calculate the mean recursively and using that recursive
mean. How? The formula is
\begin{equation}
  \label{eq:52}
  s^2 = \frac{n\left(\sum^n_1 x_i^2\right) - \left( \sum^n_1
      x_i\right)^2 } {n(n-1)}
\end{equation}
However, this method is vulnerable to the effects of round-off error
in floating-point operation. So, a better work-around (described in
Sect.4.2.2 of Donald Knuth's ``Art of Computer Programming, Vol.2''
is\footnote{\url{http://mathcentral.uregina.ca/QQ/database/QQ.09.02/carlos1.html}}.
\begin{verbatim}
M(1) = x(1), S(1) = 0
LOOP i =2 ... n
  M(i) = M(i-1) + (x(i)-M(i-1))/i
  S(i) = S(i-1) + (x(i) - M(i-1)) * (x(i) - M(i))
END

sigma = sqrt(S(n)/ (n-1)
\end{verbatim}


Now, the third problem is what if we have 2 dataset, each with its own
mean, and variance (m1,s1), (m2,s2). If we mix them together into a
single dataset, can we estimate the mean and variance of this combined
dataset?. \url{http://www.physicsforums.com/showthread.php?t=325534},
\url{http://communities.ptc.com/groups/member-showcase/blog/2010/10/07/estimating-combined-variance-from-sub-groups}. 

\section{Steady-state estimation: Balance the fluxes in the model}
\label{sec:steady_state}

Suppose that initially, we have the expected value for Cmyo (0.1$\muM$) and Cnsr
(1000$\muM$). After one second, Cmyo can be lowered, and Cnsr can be higher, or
vice versa. But total should be remained constant. This requires calcium
homeostasis, i.e. total calcium in and out across the SL should be remained
constant (which we need to calculate the total ions in and ions out). To do so,
we estimate total influx via SL, and total outflux via SL, and map them to the
ions. During this simulation, remember to set $V_m$ as dynamics, i.e. we run
I-clamp, in which we set $I_\app=0$. IMPORTANT: $I_\ncx$ is the total current of
ions which is negative. However to calculate total calcium out by NCX, we need
to multiply with (-2).

\subsection{Define parameters}

At first, we need to define what parameters should not be changed, to follow the
biological and physiolical constrains. For example, cell volume, cytosolic
volume, ionic currents. 

\subsection{Balance flux across SL}

At the first step, we block SR. We know that $[\Ca]_o=1.8$mM is fixed. Then, we
allows cytoplasm calcium to change until it reaches a steady-state value, i.e.
net flux across SL is zero. We also need to allow $V_m$ to changes so that we
can know the exact resting potential. 

If at the end, the steady-state value of $[\Ca]_\myo$ is too low or too high, we
can re-adjust it by modifying the background calcium conductance. IMPORTANT: In
such simulation, all other currents need to be enabled, and running at the
proper cell volume setting. 

At the end, we know the steady-state calcium, given the assumption that $\Ca$
across SR is balanced. But we haven't balance the flux across SR yet. It's in
the next step. The expected $[\Ca]_\myo$ at rest should be about 0.09-0.1$\muM$.

The ideal is to have a plot \verb!g_bCa! vs. \verb!Ca_myo!($\infty$).
The following pair is good enough: 
\begin{verbatim}
g_bCa = 7.70918d-4
Ca_myo = 0.93438d-01
\end{verbatim} 

\subsection{Balance flux across SR}

Now, we know the steady-state $[\Ca]_\myo$ should be. We want to know the
resting value of $[\Ca]_\sr$. The constraints is that the total calcium leak,
number of sparks, spark duration. 

We run the fixed Cmyo setting, to see with this given of Cmyo, Cnsr what is the
total $\Ca$ uptake per second. We know that total calcium leak per second is
about 1-10$\muM$, but very closed to 1$\muM$, given the constrains of sparks
rate and non-sparks events. So, we estimate \verb!Kp_nsr!, which is an unknown
parameter.

After adjusting \verb!Kp_nsr!, we run the simulation \verb!_MODE_CLOSED_CELL!
setting. Here, we balance the amount of $\Ca$ across the SR, until a proper leak
($>3000$ quarks, in which about 100 sparks, spark duration is 25ms on average
in rat). 

\subsection{Run in resting condition}


After balancing the flux across SR and SL, we can run the resting condition
\verb!_MODE_REST_CELL! setting to make sure the fluxes are balanced. 

Finally, we run the stimulus setting I-clamp or V-clamp. 

 



