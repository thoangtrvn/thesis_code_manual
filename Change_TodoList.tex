
\chapter{Updates/Plans}
\label{sec:updates}

\section{Update parameter file}
\label{sec:parameter-file}

change parameter files:
\begin{enumerate}

\item  substitute \verb!v_ryr_nj!
\begin{verbatim}
replace v_ryr_nj --> prcent_ryr (i.e. 5% or 0.05)
\end{verbatim}

\item substitute \verb!P_dhpr_nj!
\begin{verbatim}
replace P_dhpr_nj --> prcent_dhpr-nj (i.e. 10% or 0.1)
\end{verbatim}


\item substitute \verb!P_dhpr_T!
\begin{verbatim}
replace P_dhpr_T --> i_1dhpr
\end{verbatim}
and add scaling macro \verb!_USE_SCALING_MODEL_! to determine whether
scaling is used or not. 

\item add 3 parameters after \verb!t_end! and before \verb!dt_LB! (Jan, 14,
2012)
\begin{verbatim}
0.0d0         first_beat I/O (0=No / 1=Yes)
2.0d0         last_beats I/O (duration) [sec]
0.0d0         use_restart_file (0=No / 1=Yes)
\end{verbatim}
then, data mode \verb!_IO_RESTART_FILE! is removed. 
\end{enumerate}

\subsection{Affect leak model}
\label{sec:affect-leak-model}

\begin{enumerate}
\item change the name for specific membrane capacitance
\begin{verbatim}
Cm --> Csc
\end{verbatim}
as Cm will be used for whole-cell membrane capacitance.

\item change variables for buffering fractions (to discriminate with buffer
  concentration \verb!B_myoT!...)
\begin{verbatim}
B_ds --> beta_ds  
B_nsr --> beta_nsr
B_myo --> beta_myo
B_jsr --> beta_jsr
\end{verbatim}



\item change variable
\begin{verbatim}
! to indicate total ryr_nj transfer rate, not a single channel
v_ryr_nj --> v_ryr_nj_T
\end{verbatim}


\item change variable (to discriminate half-saturation constant Km vs. dissociation
  constant Kd)
\begin{verbatim}
Kd_myo --> Km_myo
Kd_jsr --> Km_jsr
\end{verbatim}

\item change volume ratio variables
\begin{verbatim}
lambda1 --> lambda_ds
lambda2 --> lambda_jsr
lambda  --> lambda_nsr
\end{verbatim}

\item change variable
\begin{verbatim}
Mbig --> NUM_STATES_SFU
\end{verbatim}

\item Automatically renames all parameters to ``parameter.conf'' so that it's
easier to read code in IDL.
\begin{verbatim}
CHARACTER(MAX_FILENAME) :: default_param_filename = 'parameters.conf'
\end{verbatim}

\item The variable \verb!ID_logfile! is replaced by \verb!log_string! and call
to \verb!write2log(log_string)!. 
\end{enumerate}

\subsection{Affect EC-coupling}
\label{sec:ec-coupling-1}

\begin{enumerate}
\item Move vv1, vv2 to \verb!Vm_dependent.inc! file, and add to the
  code via
\begin{verbatim}
INCLUDE "Vm_dependent.inc"
\end{verbatim}


\item Change the variables for random numbers
\begin{verbatim}
X     --> X_r
X_dev --> X_r_dev
\end{verbatim}


\item move \verb!sfu_state! to file: \verb!data_ecc.f95!

\item rename vv1, vv2
\begin{verbatim}
vv1 --> kva
vv2 --> kvi
\end{verbatim}


\item name change
\begin{verbatim}
isfu --> current_state_sfu --> sfu_state
isfu_dev --> current_state_sfu_dev --> sfu_state_dev
\end{verbatim}

\item change maxnk\ldots (with new algorithms)
\begin{verbatim}
maxnkv1 --> lcc_Ncol_Kv_act
maxnkv2 --> lcc_Ncol_Kv_inact
maxnkp1 --> lcc_Ncol_Kcal_inact
maxnkm1 --> lcc_Ncol_Kfree
maxnkm2 --> ryr_Ncol_Kfree
maxnkp2 --> ryr_Ncol_Kcal_act
\end{verbatim}

\item change \verb!sfu_state! 
\begin{verbatim}
sfu_state --> CRU_states
\end{verbatim}

\item change idxK, compK \ldots 
\begin{verbatim}
compKv1 --> lcc_compKv_act
compKv2 --> lcc_compKv_inact
compKp1 --> lcc_compKcal_inact
compKm1 --> lcc_compKfree
compKm2 --> ryr_compKfree
compKp2 --> ryr_compKcal_act
       
idxKv1 -->  lcc_indxKv_act
idxKv2 --> lcc_indxKv_inact
indxKm1 --> lcc_indxKfree
idxKm2 --> ryr_indxKfree
indxKp1 --> lcc_indxKcal_inact
indxKp2 --> ryr_indxKcal_act
\end{verbatim}

\item move \verb!init_GPU()! from \verb!data_global.f95! to
\verb!cuda_interop.f95!

\item move \verb!mL, mR! to the 2 new parameter files

\item rename \verb!is_Clamped! to \verb!is_being_clamped!

\item rename (to match with the similar function in spatial code) [2012-Nov-20]
\begin{verbatim}
updateSystem_2 --> update_SFU_2
updateSystem_2b --> update_SFU_2b
\end{verbatim}
\end{enumerate}

\subsection{Affect Spatial model}
\label{sec:affect-spatial-model}

\begin{enumerate}
\item Remove \verb!inv_RT_F!, the equivalent to use is \verb!F_RT!
\item Diffusion is now allows to be anisotropic at both calcium-bound
fluorescence and fluorescence with \verb!D_F! and \verb!D_CaF! are replaced by 
\begin{verbatim}
 REAL(KIND=dp), constant :: D_Fx, D_Fy, D_Fz, D_CaFx, D_CaFy, D_CaFz
\end{verbatim}

\item SFU is replaced by CRU as SFU refer to a functional group of RyRs only.
CRU refers to a fusion of RyR and LCCs. 
\begin{verbatim}
init_SFU_state --> init_CRU_state
sfu_state --> CRU_state
\end{verbatim}

\item Rename
\begin{verbatim}
some_caffeinedump_CRU_notfullyOpen --> some_caffeineBound_CRU_notfullyopen
num_timesteps_before_setting_allchannels_open --> num_timesteps_holdingchannels
\end{verbatim}

\item Rename 
\begin{verbatim}
_CRU_UNIFORM_WITH_SMALL_RYR --> _CRU_UNIFORM_WITH_ROGUE_RYR
\end{verbatim}

\item Rename [2013-Jan-29]
\begin{verbatim}
put_SFU2grid3D_homo           --> put_SFU2grid3D_uniform
put_SFU2grid3D_homo_rogue_RYR --> put_SFU2grid3D_uniform_rogue_RYR
\end{verbatim}
\end{enumerate}

% \item change parameter files: 
% \begin{verbatim}
% replace v_ryr_T --> i_1ryr (current single ryr channel)
% \end{verbatim}

% \item change $zCa$ to \verb!z_Ca!
% \begin{verbatim}
% zCa -> z_Ca
% zNa -> z_Na
% zK  -> z_K
% \end{verbatim}

% \item change zF to \verb!frdy! (Faraday constant)

% \item change zR to \verb!uniR! (universal gas constant)

%\end{enumerate}

\subsection{Affect Spatial Model with neighboring cluster}

\begin{enumerate}
  \item \verb!stride2smallRyR! is replaced by \verb!stride2rogueRyR!
  \item The macro name \verb!_ADD_SMALL_RYRS_! is replaced by a read-in value in
  the parameter file \verb!testspatialdata_file.txt!
  \begin{verbatim}
 6.0000E+0      num rogue RyR-neighbors (0,1,2,3,6)
 5.0000E+0      num RyRs in rogue cluster
  \end{verbatim}
  
  \item \verb!stride2rogueRyR! is extended and replaced by 6 variables
  \begin{verbatim}
  stride2rogueRyR_x_up, stride2rogueRyR_x_down
  stride2rogueRyR_y_left, stride2rogueRyR_y_right
  stride2rogueRyR_z_front, stride2rogueRyR_z_back
  \end{verbatim}
  The reason is that subspace is not a the same in all dimension but has the
  following dimension 15nm x 200 nm x 350nm. 
  
  \item The following code is removed from branch 2 version
  \begin{verbatim}
INTEGER, DIMENSION(:), ALLOCATABLE ::  index_CRUstate_numRyRopen  ! CRUstate index where #openRyR is 1, 2, ... N_R
INTEGER, DIMENSION(:), ALLOCATABLE ::  index_rogueRyR_numRyRopen  ! index where #openRyR is 1, 2, ... N_R_small
  \end{verbatim}
  
  
  \item \verb!dist2rogueRyR! tell the distance from
  center-CRU to the satellite RYR cluster (old-name: \verb!dist2smallRYR!,
  \verb!dist2satelliteRYRs!) [2014-Apr-18]
\end{enumerate}

\section{Plans}
\label{sec:plans}

\begin{enumerate}
  \item create the reference volume Vol$_{ref}$ which is assigned to $V_\myo^T$
  by default, and then replace 
  \begin{verbatim}
  Am_2FVmyo    ---> Am_2FV
  Am_FVmyo     ---> Am_FV
  inv_2FVmyo   ---> inv_2FV
  \end{verbatim}
\item (parameter file)  remove if not in used
\begin{verbatim}
V_theta2
V_sigma2
\end{verbatim}

\item (parameter file) remove Ej as it can be calculated from \verb!N_R!. 
\begin{verbatim}
Ej =1/(sqrt(N_R)*(sqrt(N_R)-1))((sqrt(N_R)-2)^2*4+(sqrt(N_R)-2)*4*3+4*2)
\end{verbatim}


\item Add junctional NCX  \verb!I_ncx_j! to the ECC non-spatial model
  and probably the spatial model
\begin{verbatim}
## in each CRU
I_ncx_j  = I_ncx_bar_j*(na_i^3*ca_o*EXP(0.35*V/RT_F)-na_o^3*ca_ds*EXP((0.35-1)*V/RT_F))./
         ((87500^3+na_o^3)*(1380+ca_o)*(1+0.1*EXP((0.35-1)*V/RT_F)))
J_ncx_j = -2*I_ncx_j*Am_2FVmyo
# which contribute to [Ca]_ds
\end{verbatim}

\item Remove in spatial model

\begin{verbatim}
J_efflux_total
J_refill_total
I_dhpr_total
\end{verbatim}


\item instead of using \verb!sum_reduce_double_2host()!, define a
  kernel that can be called from Fortran so that 2 kernels can run at
  the same time using 2 different stream IDs.


\item add to the spatial code
\begin{verbatim}
ts_Na_lin
ts_K_lin
\end{verbatim}
to functions
\begin{verbatim}
tscell_initialize_linear
...
\end{verbatim}


\item Consider spatial heterogeneous distribution of NCX where NCX is
  present more in junctional SR, or along the T-tubules. 


\item Implement \verb!makeKs_nj()! that work with any number of states
  for  RyR and DHPR


\item \sout{Remove 2 output set of columns RyRGate() and
  DHPRgate() from HDF files (and rerun all restart files with
  control, HF, and HForphan...)}


\item Consider moving caffeine setting to the \verb!testdata_file.txt!

\item \sout{Add the param file name in the summary\_data.txt so that we
don't have to search for correc tconfiguration file name. (add to the first
line)}

\item Implement print out to restart file \verb!datamode6_init!,
\verb!datamode6_print!, \verb!datamode6_print_last!.

\item Change in spatial code to reflect better name (flux from SR to MYO should
be flux from internal source (either NSR or subspace) to MYO)
\begin{verbatim}
J_sr2myo ---> J_int2myo
\end{verbatim}


\end{enumerate}

