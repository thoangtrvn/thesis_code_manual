\chapter{Data I/O}
\label{chap:Data_IO}

\begin{lstlisting}
CALL data_io_init()

CALL data_io_write()

CALL data_io_finalize()
\end{lstlisting}
which will call appropriate I/O functions depending on the setting of
macro values. 

There are a number of modes
\begin{verbatim}
#define _IO_NONE  0  
#define _IO_TEST  1
#define _IO_2_HDF  2
#define _IO_2_ASCII  3
#define _IO_2_SILO_3D 4 
#define _IO_2_SILO_SLICE 5
#define _IO_RESTART_FILE 6 
#define _IO_2_HDF2 7

#define _DATAMODE_ _IO_RESTART_FILE
\end{verbatim}
The description for each mode is given as follows
\begin{verbatim}
  !   0 = no I/O (only summary file is written)
  !   1 = testing purpose (we can freely modify to print out 
  !        (what we want in ASCII format)
  !   2 = (spatial or non-spatial) HDF files
  !   3 = text files (all)
  !   4,5 = spatial output
  !   6 = write out restart file and the last second
  !   7 = (non-spatial) multiple datasets/file (time-Vm, Cads, Cajsr...)
  !   8 = (spatial) HDF whole-cell
NOTE: restart file is written out in all cases, except _IO_NONE  
\end{verbatim}
All modes print out the restart files. In mode \verb!_IO_RESTART_FILES!, we also
print out the last second.

We have two time trackers:
\begin{verbatim}
time_prev1 = to keep track when to write data out
time_prev2 = to keep track time for calculating integrate flux
time_prev_fb = to keep track when to write data out in first-second
\end{verbatim}


\section{Leak code}
\label{sec:IO_leak}

It still uses 
\begin{verbatim}
#if _DATAMODE_ == 1
    CALL datamode1_print()
#elif _DATAMODE_ == 2
    CALL datamode2_print()
#elif _DATAMODE_ == 3
    CALL datamode3_print()
#endif
\end{verbatim}

\section{ECC code}
\label{sec:IO_ecc}

Even though there are many options to choose the output format (as
described in the below subsections), there are another two options to choose how
often you want to do I/O
\begin{enumerate}
  \item \verb!_METHOD1!: write every \verb!x! (sec) with \verb!x! read from
  parameter file
  \item \verb!_METHOD2!: write after a given number of iterations (default: 100)
\end{enumerate}

\subsection{\_IO\_2\_HDF}
\label{sec:ECC_HDF}

Old versions: data is organized in a single data set ``/LEAK'' with a large
number of columns. The functions for \verb!_IO_2_HDF!, it uses
\begin{verbatim}
datamode2_init()
datamode2_print()
  CALL hdf_open_simple(file_out, NSFU, CHUNK_SIZE, CHUNKS_PER_FILE, 
                   file_struct, FIELDS_PER_RECORD)
  CALL hdf_write_simple(file_struct, data_buffer, CHUNK_SIZE, 
                   FIELDS_PER_RECORD)
  CALL hdf_spark_close(file_struct)

datamode2_print_last()       
datamode2_close()
\end{verbatim}

\subsection{\_IO\_2\_HDF2}
\label{sec:ECC_HDF2}

New version: data is organized into a number of datasets
\begin{enumerate}
  \item ``/general''
  \item ``/sfu-state''
  \item ``/cds''
  \item ``/cjsr''
  \item ``/cacm''
\end{enumerate}
The functions for \verb!_IO_2_HDF2!, it uses
\begin{verbatim} 
datamode7_init()
datamode7_print()
  CALL hdf_ecc_open(file_out, NSFU, CHUNK_SIZE, 
                       CHUNKS_PER_FILE, file_struct, FIELDS_PER_RECORD)
  CALL hdf_ecc_write(file_struct, data_buffer, CHUNK_SIZE, 
                       FIELDS_PER_RECORD, NSFU)
  CALL hdf_ecc_close(file_struct)

datamode7_close()
\end{verbatim}



\section{Spatial code}
\label{sec:IO_spatial}

Here, we want to make sure all files have the same number of time points, so we
decided to divide files into a number of categories:
\begin{enumerate}
  \item \verb!file_struct! (HDFFILE): non-spatial or regular data (Vm, Cads,
  Cajsr, etc.)
  \item \verb!file_struct_cmyo! (HDFFILE2, HDFFILE3)
  \item \verb!file_struct_caf! (HDFFILE2, HDFFILE3)
  \item \verb!file_struct_cnsr! (HDFFILE2, HDFFILE3)
  \item \verb!file_struct_Incx! (HDFFILE2, HDFFILE3)
  \item (future) \ldots 
\end{enumerate}
\begin{verbatim}
  TYPE(HDFFILE) :: file_struct
#if _DATAMODE_ == _IO_2_HDF  
  TYPE(HDFFILE2) :: file_struct_cmyo, file_struct_CaF, file_struct_cnsr
#elif _DATAMODE_ == _IO_2_HDF3D
  TYPE(HDFFILE3) :: file_struct_cmyo, file_struct_CaF, file_struct_cnsr
  TYPE(HDFFILE3) :: file_struct_Incx
#endif
\end{verbatim}
with the description of each type as follows
\begin{verbatim}
  TYPE HDFFILE !regular data
     INTEGER(HID_T) :: file_ID
     INTEGER(HID_T), DIMENSION(MAX_NUM_HANDLE) :: arr_dspace ! (data) space
     INTEGER ::  num_dspace
     INTEGER(HID_T), DIMENSION(MAX_NUM_HANDLE) :: arr_dset ! dataset
     INTEGER :: num_dset
     INTEGER(HID_T), DIMENSION(MAX_NUM_HANDLE) :: arr_plist !property list
     INTEGER :: num_plist
  END TYPE HDFFILE
\end{verbatim}

Each file has one chunk (\verb!CHUNKS_PER_FILE=1!). Each chunk is a table of
size \verb!CHUNK_SIZE! x \verb!FIELDS_PER_RECORD!, with
\begin{verbatim}
    CHUNK_SIZE = 20000
    CHUNKS_PER_FILE = 1 ! set it to one for now
    NUM_STATES_NJ = mR * mL
    NUM_FIELDS4CURRENTS = 11
    !2 columns = 1 for time + 1 for Vm
    !first NSFU = sfu_state
    !second NSFU = cds
    !third NSFU = cjsr
    FIELDS_PER_RECORD = 2 + NSFU + NUM_STATES_NJ + NUM_FIELDS4CURRENTS + 2*NSFU
\end{verbatim}
 
\begin{verbatim}
  TYPE HDFFILE2 !to deal with spatial data
     INTEGER(HID_T) :: file_ID
     INTEGER(HID_T), DIMENSION(MAX_NUM_HANDLE) :: arr_dspace ! (data) space
     INTEGER ::  num_dspace
     !first = vlayer
     !second= hlayer
     INTEGER(HID_T), DIMENSION(2,MAX_NUM_HANDLE,MAX_NUM_GROUPS) :: arr_dset ! dataset (1:2,dataset_id, group_id)
     INTEGER, DIMENSION(MAX_NUM_HANDLE) :: num_dset !number of dset in each group
     INTEGER(HID_T), DIMENSION(MAX_NUM_HANDLE) :: arr_plist !property list
     INTEGER :: num_plist
     INTEGER(HID_T), DIMENSION(MAX_NUM_GROUPS) :: arr_group ! group
     INTEGER :: num_group
     !the zero-th element keep the number of elements in the array
     REAL(KIND=dp), DIMENSION(MAX_NUM_HANDLE*MAX_NUM_GROUPS) :: time_data
  END TYPE HDFFILE2
\end{verbatim}

\begin{verbatim}
  TYPE HDFFILE3 !spatial whole-cell data
     INTEGER(HID_T) :: file_ID
     INTEGER(HID_T), DIMENSION(MAX_NUM_HANDLE) :: arr_dspace ! (data) space
     INTEGER ::  num_dspace
     INTEGER(HID_T), DIMENSION(MAX_NUM_HANDLE) :: arr_dset ! dataset
     INTEGER :: num_dset
     INTEGER(HID_T), DIMENSION(MAX_NUM_HANDLE) :: arr_plist !property list
     INTEGER :: num_plist
     REAL(KIND=dp), DIMENSION(MAX_NUM_HANDLE) :: time_data
  END TYPE HDFFILE3
\end{verbatim}

\subsection{\_IO\_2\_HDF}

We print out 2 layers (longitudinal and transversal direction) at each Z-lines.
The mode is \verb!_IO_2_HDF2!, it uses
\begin{verbatim}
datamode2_init_2()
datamode2_print_2()
    CALL get_new_cmyo_filename(hdf_cmyo_filename, hdf_spatial_fileindex)
    CALL hdf_spatial_open(file_struct_cmyo, hdf_cmyo_filename, num_vlayers, num_hlayers, &
         outerDIMX, outerDIMY, outerDIMZ)
    CALL get_cnsr_filename(hdf_cnsr_filename, hdf_spatial_fileindex)
    CALL hdf_spatial_open(file_struct_cnsr, hdf_cnsr_filename, num_vlayers, num_hlayers, &
         outerDIMX, outerDIMY, outerDIMZ)
  #if _USE_FLUO_ != _NO
    CALL get_CaF_filename(hdf_CaF_filename, hdf_spatial_fileindex)
    CALL hdf_spatial_open(file_struct_CaF, hdf_CaF_filename, num_vlayers, num_hlayers, &
         outerDIMX, outerDIMY, outerDIMZ)
  #endif
    CALL hdf_spatial_write(file_struct_cmyo, ts_Ca_myo, &
         num_vlayers, idx_vlayers, num_hlayers, idx_hlayers, &
         outerDIMX, outerDIMY, outerDIMZ, time, file_is_full)

    CALL hdf_spatial_write(file_struct_cnsr, ts_Ca_nsr, &
         num_vlayers, idx_vlayers, num_hlayers, idx_hlayers, &
         outerDIMX, outerDIMY, outerDIMZ, time, file_is_full)

    CALL hdf_open_simple(file_out, NSFU, CHUNK_SIZE, CHUNKS_PER_FILE, file_struct, FIELDS_PER_RECORD)
    CALL hdf_write_simple(file_struct, data_buffer, CHUNK_SIZE, FIELDS_PER_RECORD)
    CALL hdf_spark_close(file_struct)
  
datamode2_close_2()
\end{verbatim}
Given the number of time points is CHUNK\_SIZE=10,000; we divide the
spatial data in HDF5 into 100 groups, each group has 2x100 datasets (first 100
belongs to cmyo-vertical, second 100 belongs to cmyo-horizontal).

\begin{framed}
Using 10,000 dataset per file, it gives 15GB of data. So, we suggest using
300 dataset per file, which gives 500MB per file.
\end{framed}
Here, we use two different file indices, one for HDF5 spatial data
\verb!hdf_spatial_fileindex!, and one for HDF5 regular data \verb!file_index!. 

The HDF4 spatial code is organized as follows:
\begin{enumerate}
  \item 4 groups. Name of groups /group-001, /group-002\ldots, each group has 2x100
  datasets.
  \item the first 100 dataset 
\end{enumerate}

For cmyo spatial data, we do
\begin{verbatim}
hdf_spatial_fileindex = 0 !deal with spatial data
hdf_groupindex = 0  ! deal with group
hdf_datasetindex_ingroup = 0 ! deal with dataset in the current group
!automatically increase file index and return the file name
CALL get_new_cmyo_filename(hdf_cmyo_filename, hdf_spatial_fileindex)
CALL hdf_spatial_open(file_struct_cmyo, hdf_cmyo_filename, hdf_groupindex, num_vlayers, num_hlayers, &
     outerDIMX, outerDIMY, outerDIMZ)
\end{verbatim}

In \verb!hdf_spatial_open!, it does
\begin{enumerate}
  \item We create the file
\begin{verbatim}
CALL h5fcreate_f(TRIM(filename), H5F_ACC_TRUNC_F, file_ID, hdf_error, &
         H5P_DEFAULT_F, H5P_DEFAULT_F)
file_struct%file_ID = file_ID
\end{verbatim}

%\item We create attribute for root group (/) to store some information

\item We create the group
\begin{verbatim}
    !create the first group
    WRITE(group_name, "('/group-', i3.3)") group_index
    CALL h5gcreate_f(file_ID, TRIM(group_name), hdf_error, 0)
    file_struct%num_group = file_struct%num_group + 1
\end{verbatim}

\end{enumerate}

Initialy the variable \verb!file_is_full=FALSE!. Then, when we write the data 

\begin{enumerate}
  \item if file is full, then
	\begin{enumerate}
	  \item write time-data to dataset ``time'' belong to root group
	  \item close everything
	\end{enumerate}
  \item if not
  \begin{enumerate}
    \item check if group is full, if yes, create new group
    \item create new dataset: 2 for each group ``/hdataset-'', and ``/vdataset-''
    \item check if file is full to update \verb!file_is_full!.
   \end{enumerate}  
\end{enumerate}


\begin{verbatim}
  !! keep the indices of which layers that have the CRUs
  INTEGER :: num_vlayers, num_hlayers
  INTEGER, DIMENSION(MAX_NUM_LAYERS) :: idx_vlayers, idx_hlayers
\end{verbatim}

\subsection{\_IO\_2\_HDF2}

???

\subsection{\_IO\_2\_HDF3D}

We print out the whole cell, \verb!_IO_2_HDF3D!, calcium concentration in the
myoplasm Camyo, network SR Cansr; and possibly calcium-bound fluorescence in the
myoplasm CaF.
\begin{verbatim}
datamode8_init()
datamode8_print()
       CALL hdf_spatial_close_2(file_struct_cmyo)
       CALL get_new_cmyo_filename(hdf_cmyo_filename, hdf_spatial_fileindex)
       CALL hdf_spatial_open_2(file_struct_cmyo, hdf_cmyo_filename,  &
            outerDIMX, outerDIMY, outerDIMZ)

       CALL hdf_spatial_close_2(file_struct_cnsr)
       CALL get_cnsr_filename(hdf_cnsr_filename, hdf_spatial_fileindex)
       CALL hdf_spatial_open_2(file_struct_cnsr, hdf_cnsr_filename,  &
            outerDIMX, outerDIMY, outerDIMZ)
datamode8_close()
\end{verbatim}

In this mode, to determine what to write out, a few more macros have been added
to macro.h
\begin{verbatim}
#define _WRITE_CMYO_ _YES
#define _WRITE_CNSR_ _YES
#define _WRITE_BOUND_FLUO_ _YES
#define _WRITE_I_NCX_ _NO
\end{verbatim}

For code before Oct 31, 2012, non-spatial data are written out using old format
like \verb!_IO_2_HDF! in ECC code (Sect.\ref{sec:ECC_HDF}).

For code since Nov, 1st, 2012, non-spatial data are written out using new format
like \verb!_IO_2_HDF2! in ECC code (Sect.\ref{sec:ECC_HDF2}) with some changes
\begin{enumerate}
  \item column 1 = time
  \item column 2 = Vm
  \item next \verb!NUM_FIELDS4CURRENTS! columns = gating parameters
  \item next NSFU columns = index of CRU states
  \item next NSFU columns = Cads
  \item next NSFU columns = Cajsr
\end{enumerate}

\subsection{\_IO\_2\_SILO\_3D}

Print out the whole 3D data to SILO format
\begin{verbatim}
datamode4_init()
datamode4_print()

datamode4_close()
\end{verbatim}

\subsection{\_IO\_2\_SILO\_SLICE}

Print out a single slice (indicated by \verb!slice_index=!) to SILO format
\begin{verbatim}
datamode5_init()
datamode5_print()

datamode5_close()
\end{verbatim}